\documentclass[11pt]{article}
%% Packages
\usepackage[T1]{fontenc}
\usepackage{amssymb}
\usepackage{amsmath}
\usepackage{amsthm}
\usepackage{amsfonts}
\usepackage{mathtools}
\usepackage{xspace}
\usepackage{bm}
\usepackage{nicefrac}
\usepackage{commath}
\usepackage{bbm}
\usepackage{boxedminipage}
\usepackage{xparse}
\usepackage{xcolor}
\usepackage{float}
\usepackage{multirow}
\usepackage{graphicx}
\usepackage{caption}
\usepackage{subcaption}
\usepackage{ifthen}
\usepackage{algpseudocode}
\usepackage{proba} 
\usepackage{colortbl} 
%\usepackage{upgreek}
%\usepackage{paralist}
\usepackage{enumitem}
%\usepackage{times}

\usepackage{tikz}
\usetikzlibrary{positioning}
\usetikzlibrary{fit}
\usetikzlibrary{calc}
\usetikzlibrary{backgrounds}
\usetikzlibrary{shapes}
\usetikzlibrary{patterns}
\usetikzlibrary{matrix}

\usepackage{geometry}

%\usepackage[pdftex,bookmarks=true,pdfstartview=FitH,colorlinks,linkcolor=blue,filecolor=blue,citecolor=blue,urlcolor=blue]{hyperref}
%    \urlstyle{sf}

%\usepackage{enumitem}


%% Document Design Choices
%\setlength{\parindent}{0pt}
%\setlength{\parskip}{10pt}
%\renewcommand{\labelitemi}{\ensuremath{\circ}}
%\geometry{left=1in, right=1in, top=1in, bottom=1in}


%% Marking
\newcommand\TODO[1]{{\color{red}{#1}}\xspace}
\newcommand\change[1]{{\color{red}\underline{#1}}\xspace}
\newcommand{\commentdg}[1]{{\color{blue}{#1}}\xspace}
\newcommand\commentop[1]{{\color{red}{#1}}\xspace}


%% Theorems and Environments
\newtheorem{conjecture}{Conjecture}
%\newtheorem{theorem}{Theorem}
%\newtheorem{importedtheorem}{Imported Theorem}
%\newtheorem{informaltheorem}{Informal Theorem}
%\newtheorem{definition}{Definition}
%\newtheorem{lemma}{Lemma}
\newtheorem{claim}{Claim}
%\newtheorem{corollary}[theorem]{Corollary}
%\newtheorem{observation}{Observation}
%\newtheorem{property}{Property}


\newenvironment{boxedalgo}
  {\begin{center}\begin{boxedminipage}{1\linewidth}}
  {\end{boxedminipage}\end{center}}

\newenvironment{inlinealgo}[1]
  {\noindent\hrulefill\\\noindent{\bf {#1}}}
  {\hrulefill}


%% References
\newcommand{\namedref}[2]{\hyperref[#2]{#1~\ref*{#2}}\xspace}
\newcommand{\lemmaref}[1]{\namedref{Lemma}{lem:#1}}
\newcommand{\theoremref}[1]{\namedref{Theorem}{thm:#1}}
\newcommand{\inftheoremref}[1]{\namedref{Informal Theorem}{infthm:#1}}
\newcommand{\claimref}[1]{\namedref{Claim}{clm:#1}}
\newcommand{\corollaryref}[1]{\namedref{Corollary}{corol:#1}}
\newcommand{\figureref}[1]{\namedref{Figure}{fig:#1}}
\newcommand{\equationref}[1]{\namedref{Equation}{eq:#1}}
\newcommand{\defref}[1]{\namedref{Definition}{def:#1}}
\newcommand{\observationref}[1]{\namedref{Observation}{obs:#1}}
\newcommand{\procedureref}[1]{\namedref{Procedure}{proc:#1}}
\newcommand{\importedtheoremref}[1]{\namedref{Imported Theorem}{impthm:#1}}
\newcommand{\informaltheoremref}[1]{\namedref{Informal Theorem}{infthm:#1}}
\newcommand{\sectionref}[1]{\namedref{Section}{sec:#1}}
\newcommand{\appendixref}[1]{\namedref{Appendix}{app:#1}}
\newcommand{\propertyref}[1]{\namedref{Property}{prop:#1}}


%% English
\newcommand{\ie}{\text{i.e.}\xspace}
\newcommand{\st}{\text{s.t.}\xspace}
\newcommand{\etal}{\text{et al.}\xspace}
\newcommand{\naive}{\text{na\"ive}\xspace}
\newcommand{\wrt}{\text{w.r.t.}\xspace}
\newcommand{\whp}{\text{w.h.p.}\xspace}
\newcommand{\resp}{\text{resp.}\xspace}
\newcommand{\eg}{\text{e.g.}\xspace}
\newcommand{\cf}{\text{cf.}\xspace}
\newcommand{\visavis}{\text{vis-\`a-vis}\xspace}

%% Authors
\newcommand{\damgard}{Damg{\aa}rd\xspace}


%% Crypto
\def\alice{{{Alice}}\xspace}
\def\bob{{{Bob}}\xspace}
\def\eve{{{Eve}}\xspace}
\newcommand{\poly}{\ensuremath{\mathrm{poly}}\xspace}
\newcommand{\polylog}{\ensuremath{\mathrm{polylog}\;}\xspace}
\newcommand{\negl}{\ensuremath{\mathsf{negl}}\xspace}
\newcommand{\secpar}{\ensuremath{{\lambda}}\xspace}
\newcommand{\zo}{\ensuremath{{\{0,1\}}}\xspace}

  %% Functionalities
  \newcommand{\func}[1]{\ensuremath{\cF_{\mathsf{#1}}}\xspace}
  \newcommand{\fcom}[0]{\func{com}}
  \newcommand{\fot}[0]{\func{ot}}
  
  %% Protocols
  % \newcommand{\prot}[1]{\ensuremath{\pi_{\mathsf{#1}}}\xspace}


%% Graph Theory
\newcommand{\ancest}[0]{\mathsf{Ancestors}}
\newcommand{\sibling}[0]{\mathsf{Siblings}}
\newcommand{\parent}[0]{\mathsf{parent}}
\newcommand{\leaves}[0]{\mathsf{leaves}}


%% Math Symbols
\renewcommand{\o}[0]{\ensuremath{\circ}\xspace}
\newcommand{\dist}[1]{\ensuremath{\left\langle{#1}\right\rangle}\xspace}
%\newcommand{\prob}[2]{\ensuremath{\dist{{#1}_1, \dotsc, {#1}_{#2}}}\xspace}
\newcommand{\ip}[2]{\ensuremath{\left\langle{#1},{#2}\right\rangle}\xspace}
\renewcommand{\vec}[1]{\ensuremath{\mathbf{#1}}\xspace}
\newcommand{\concat}[0]{\ensuremath{\circ}\xspace}
\newcommand{\nin}[0]{\ensuremath{\not\in}\xspace}
\newcommand{\xor}[0]{\ensuremath{\oplus}\xspace}
\newcommand{\rv}[1]{\ensuremath{\mathbf{#1}}\xspace}
\newcommand{\p}[1]{\ensuremath{^{{\left(#1\right)}}}\xspace}
\newcommand{\argmax}[0]{\ensuremath{\mathop{\mathrm{argmax}}~}\xspace}
\newcommand{\argmin}[0]{\ensuremath{\mathop{\mathrm{argmin}}~}\xspace}
\newcommand{\me}{\mathrm{e}}
\NewDocumentCommand\mathstack{>{\SplitList{;}}m}
  {\ensuremath{
    \begin{smallmatrix}
      \ProcessList{#1}{ \insertone }    
    \end{smallmatrix}
  }}
\newcommand{\insertone}[1]{\ensuremath{#1}\\}
\newcommand{\tuple}[1]{\ensuremath{{\left\langle{#1}\right\rangle}}\xspace}

%\newcommand{\argmax}{\operatornamewithlimits{argmax}}
%\DeclareMathOperator*{\argmax}{arg\,max}

  
  %% General
  \newcommand{\ceil}[1]{\ensuremath{\left\lceil{#1}\right\rceil}\xspace}
  \newcommand{\floor}[1]{\ensuremath{\left\lfloor{#1}\right\rfloor}\xspace}
  %\newcommand{\abs}[1]{\ensuremath{\left\vert{#1}\right\vert}\xspace}
  \newcommand{\lone}[1]{\ensuremath{\left\vert{#1}\right\vert}\xspace}
  \newcommand{\spnorm}[1]{\ensuremath{\left\Vert{#1}\right\Vert}\xspace}

  %% Renamed Symbols
  \newcommand{\eps}[0]{\varepsilon}
  \let\epsilon\eps
  %\let\phi\varphi
  
  %% Cal Alphabets
  \newcommand{\cA}{\ensuremath{{\mathcal A}}\xspace}
  \newcommand{\cB}{\ensuremath{{\mathcal B}}\xspace}
  \newcommand{\cC}{\ensuremath{{\mathcal C}}\xspace}
  \newcommand{\cD}{\ensuremath{{\mathcal D}}\xspace}
  \newcommand{\cE}{\ensuremath{{\mathcal E}}\xspace}
  \newcommand{\cF}{\ensuremath{{\mathcal F}}\xspace}
  \newcommand{\cG}{\ensuremath{{\mathcal G}}\xspace}
  \newcommand{\cH}{\ensuremath{{\mathcal H}}\xspace}
  \newcommand{\cI}{\ensuremath{{\mathcal I}}\xspace}
  \newcommand{\cK}{\ensuremath{{\mathcal K}}\xspace}
  \newcommand{\cL}{\ensuremath{{\mathcal L}}\xspace}
  \newcommand{\cM}{\ensuremath{{\mathcal M}}\xspace}
  \newcommand{\cN}{\ensuremath{{\mathcal N}}\xspace}
  \newcommand{\cO}{\ensuremath{{\mathcal O}}\xspace}
  \newcommand{\cP}{\ensuremath{{\mathcal P}}\xspace}
  \newcommand{\cQ}{\ensuremath{{\mathcal Q}}\xspace}
  \newcommand{\cR}{\ensuremath{{\mathcal R}}\xspace}
  \newcommand{\cS}{\ensuremath{{\mathcal S}}\xspace}
  \newcommand{\cT}{\ensuremath{{\mathcal T}}\xspace}
  \newcommand{\cU}{\ensuremath{{\mathcal U}}\xspace}
  \newcommand{\cV}{\ensuremath{{\mathcal V}}\xspace}
  \newcommand{\cW}{\ensuremath{{\mathcal W}}\xspace}
  \newcommand{\cX}{\ensuremath{{\mathcal X}}\xspace}
  \newcommand{\cY}{\ensuremath{{\mathcal Y}}\xspace}
  \newcommand{\cZ}{\ensuremath{{\mathcal Z}}\xspace}
  
  %% Bold Alphabets
  \newcommand{\bA}{\ensuremath{{\mathbf A}}\xspace}
  \newcommand{\bB}{\ensuremath{{\mathbf B}}\xspace}
  \newcommand{\bC}{\ensuremath{{\mathbf C}}\xspace}
  \newcommand{\bD}{\ensuremath{{\mathbf D}}\xspace}
  \newcommand{\bE}{\ensuremath{{\mathbf E}}\xspace}
  \newcommand{\bF}{\ensuremath{{\mathbf F}}\xspace}
  \newcommand{\bG}{\ensuremath{{\mathbf G}}\xspace}
  \newcommand{\bQ}{\ensuremath{{\mathbf Q}}\xspace}
  \newcommand{\bU}{\ensuremath{{\mathbf U}}\xspace}
  \newcommand{\bV}{\ensuremath{{\mathbf V}}\xspace}
  \newcommand{\bX}{\ensuremath{{\mathbf X}}\xspace}
  \newcommand{\bY}{\ensuremath{{\mathbf Y}}\xspace}
  \newcommand{\br}{\ensuremath{{\mathbf r}}\xspace}
  
  %% Black-board Bold Alphabets
  \newcommand{\bbA}{\ensuremath{{\mathbb A}}\xspace}
  \newcommand{\bbB}{\ensuremath{{\mathbb B}}\xspace}
  \newcommand{\bbC}{\ensuremath{{\mathbb C}}\xspace}
  \newcommand{\bbD}{\ensuremath{{\mathbb D}}\xspace}
  \newcommand{\bbE}{\ensuremath{{\mathbb E}}\xspace}
  \newcommand{\bbF}{\ensuremath{{\mathbb F}}\xspace}
  \newcommand{\bbG}{\ensuremath{{\mathbb G}}\xspace}
  \newcommand{\bbH}{\ensuremath{{\mathbb H}}\xspace}
  \newcommand{\bbI}{\ensuremath{{\mathbb I}}\xspace}
  \newcommand{\bbJ}{\ensuremath{{\mathbb J}}\xspace}
  \newcommand{\bbK}{\ensuremath{{\mathbb K}}\xspace}
  \newcommand{\bbN}{\ensuremath{{\mathbb N}}\xspace}
  \newcommand{\bbO}{\ensuremath{{\mathbb O}}\xspace}
  \newcommand{\bbP}{\ensuremath{{\mathbb P}}\xspace}
  \newcommand{\bbQ}{\ensuremath{{\mathbb Q}}\xspace}
  \newcommand{\bbR}{\ensuremath{{\mathbb R}}\xspace}
  \newcommand{\bbS}{\ensuremath{{\mathbb S}}\xspace}
  \newcommand{\bbT}{\ensuremath{{\mathbb T}}\xspace}
  \newcommand{\bbU}{\ensuremath{{\mathbb U}}\xspace}
  \newcommand{\bbV}{\ensuremath{{\mathbb V}}\xspace}
  \newcommand{\bbZ}{\ensuremath{{\mathbb Z}}\xspace}
  
  %% Fraktur Alphabets
  \newcommand{\fP}{\ensuremath{{\mathfrak P}}\xspace}
  \newcommand{\fR}{\ensuremath{{\mathfrak R}}\xspace}
  \newcommand{\fX}{\ensuremath{{\mathfrak X}}\xspace}
  
  %% Overline Aphabets
  \newcommand{\op}{\ensuremath{{\overline p}}\xspace}
  
  %% Hat Aphabets
  \newcommand{\hf}{\ensuremath{{\widehat f}}\xspace}
  \newcommand{\hg}{\ensuremath{{\widehat g}}\xspace}
  
  %% Fractions
  \newcommand{\half}{\ensuremath{\frac12}\xspace}
  
  %% Set
  \newcommand{\comp}[1]{\ensuremath{\overline{{#1}}}\xspace}
  
  %% Models
  \newcommand{\defeq}[0]{\ensuremath{{\;\vcentcolon=\;}}\xspace}
  \newcommand{\eqdef}[0]{\ensuremath{{\;=\vcentcolon\;}}\xspace}
  \newcommand{\entails}[0]{\ensuremath{{\;\models\;}}\xspace}
  
  %% Matrix
  \newcommand{\tran}[0]{\ensuremath{^{\mathsf{T}}}\xspace}
  
  %% Probability and Distributions
  \newcommand{\event}[1]{\ensuremath{\mathsf{#1}}\xspace}
  \newcommand{\supp}[0]{\ensuremath{\mathsf{Supp}}\xspace}
  \let\pr\prob
  %\newcommand{\pr}[0]{\mathop{\mathrm{Pr}}\xspace}
  \newcommand{\var}[0]{\mathop{\mathrm{Var}}\xspace}
  \newcommand{\cov}[0]{\mathop{\mathrm{Cov}}\xspace}
  %\renewcommand{\Pr}[0]{\mathrm{\generateerror}\xspace}
  %\newcommand{\E}[0]{\mathop{\bbE}\xspace}
  \newcommand{\getsr}[0]{\mathbin{\stackrel{\mbox{\,\tiny \$}}{\gets}}}
  \newcommand{\drawn}{\ensuremath{\sim}\xspace}
  %\newcommand{\sd}[2]{\ensuremath{\mathbf{\Delta}\left({#1},{#2}\right)}\xspace}
  \newcommand{\sd}[2]{\ensuremath{\mathrm{SD}\left({#1},{#2}\right)}\xspace}
  \newcommand{\hd}{\ensuremath{\mathrm{HD}}\xspace}
  \newcommand{\kl}[2]{\ensuremath{\mathrm{D_{KL}}\left({#1},{#2}\right)}\xspace}
  \newcommand{\iid}[0]{\text{i.i.d.}\xspace}
  \newcommand{\ent}[0]{\ensuremath{\mathrm{H}}\xspace}
  
    %% Random Variables
    \newcommand{\rvA}{\rv{A}}
    \newcommand{\rvB}{\rv{B}}
    \newcommand{\rvC}{\rv{C}}
    \newcommand{\rvD}{\rv{D}}
    \newcommand{\rvG}{\rv{G}}
    \newcommand{\rvP}{\rv{P}}
    \newcommand{\rvQ}{\rv{Q}}
    \newcommand{\rvR}{\rv{R}}
    \newcommand{\rvS}{\rv{S}}
    \newcommand{\rvT}{\rv{T}}
    \newcommand{\rvU}{\rv{U}}
    \newcommand{\rvV}{\rv{V}}
    \newcommand{\rvW}{\rv{W}}
    \newcommand{\rvX}{\rv{X}}
    \newcommand{\rvY}{\rv{Y}}
    \newcommand{\rvZ}{\rv{Z}}
    
    \newcommand{\rvm}{\rv{m}}
    \newcommand{\rvr}{\rv{r}}
    \newcommand{\rvx}{\rv{x}}
    \newcommand{\rvy}{\rv{y}}
    \newcommand{\rvc}{\rv{c}}

  %% Binary Operators
  \newcommand{\band}[0]{\ensuremath{~\wedge~}\xspace}
  \newcommand{\bor}[0]{\ensuremath{~\vee~}\xspace}
  \let\leq\leqslant
  \let\le\leqslant
  \let\geq\geqslant
  \let\ge\geqslant
  
  %% Combinatorial
  \renewcommand\choose[2]{\ensuremath{{
    \left(\begin{matrix}
      {#1}\\
      {#2}
    \end{matrix}\right)}}\xspace}
  \newcommand\smallchoose[2]{\ensuremath{{
    \left(\begin{smallmatrix}
      {#1}\\
      {#2}
    \end{smallmatrix}\right)}}\xspace}
  \newcommand{\fallfact}[2]{\ensuremath{{#1}_{\left({#2}\right)}}\xspace}
  %\newcommand{\fact}[1]{\ensuremath{{#1}!}\xspace}
  
  %% Set Operations
  \newcommand{\union}[0]{\ensuremath{\cup}\xspace}
  \newcommand{\intersect}[0]{\ensuremath{\cap}\xspace}
  \newcommand{\setdiff}[0]{\ensuremath{\Delta}\xspace}
  

%% Algorithms, Predicates
\newcommand{\pred}[1]{\ensuremath{\mathsf{#1}}\xspace}


%% Local Terms
\newcommand{\mycite}[1]{{\color{brown}{{#1}}}\xspace}
\newcommand{\wt}[0]{\ensuremath{\mathsf{wt}}\xspace}
\newcommand{\Inf}[0]{\ensuremath{\pred{Inf}}\xspace}
%\newcommand{\Var}[0]{\ensuremath{\pred{Var}}\xspace}
\newcommand{\mac}[0]{\pred{Mac}} 
\newcommand{\sk}[0]{\pred{sk}} 
\newcommand{\pk}[0]{\pred{pk}} 
\newcommand{\gen}[0]{\pred{Gen}} 
\newcommand{\enc}[0]{\pred{Enc}} 
\newcommand{\dec}[0]{\pred{Dec}} 
\newcommand{\SK}[0]{\pred{sk}} 
\newcommand{\capprox}[0]{\approx\p c}
\renewcommand{\tag}[0]{\pred{Tag}} 
\newcommand{\sign}[0]{\pred{Sign}} 
\newcommand{\ver}[0]{\pred{Ver}} 
\newcommand{\simu}[0]{\pred{Sim}} 
\newcommand{\ball}[0]{\pred{Ball}\xspace}
\newcommand{\vol}[0]{\pred{Vol}\xspace}


\newcommand{\pout}[0]{\ensuremath{\p{\text{out}}}\xspace}
\newcommand{\pin}[0]{\ensuremath{\p{\text{in}}}\xspace}

\newcommand{\Ext}[0]{\ensuremath{\pred{Ext}}\xspace}
\newcommand{\inv}[0]{\pred{inv}}


\usepackage{fancyhdr}   
\pagestyle{fancy}      
\lhead{CS 355, FALL 2020}               
\rhead{Name: Christopher Cohen}

\usepackage[strict]{changepage}  
\newcommand{\nextoddpage}{\checkoddpage\ifoddpage{\ \newpage\ \newpage}\else{\ \newpage}\fi}  


\begin{document}

\title{Homework 1}

\date{}

\maketitle 

\thispagestyle{fancy}  
\pagestyle{fancy}      




\begin{enumerate}
%%%%%%%%%%%%%%%%%%%%%%%%%%%%%%%%%%%%%%%%%%%%%%%%%%
%%%%%%%%%%%% PROBLEM 1 %%%%%%%%%%%%%%%%%%%%%%%%%%%%
%%%%%%%%%%%%%%%%%%%%%%%%%%%%%%%%%%%%%%%%%%%%%%%%%%%
\item {\bfseries Estimating logarithm function.} For $x\in[0,1),$ we shall use the identity that 
  $$\ln(1-x) = -x - \frac{x^2}2 - \frac{x^3}3 - \dotsi.$$
\begin{enumerate} 
    \item {\bfseries (5 points)} Prove that $\ln(1-x) \leq - x - \frac{x^2}2.$
    \newline
    %%% ANSWER %%%
    {\bfseries
      \newline
      \newline
      For all $x\in[0,1)$, $-x$ is negative, $ -\frac{x^2}2$ is negative, $ -\frac{x^3}3$ is negative, and so on. This means that the estimation for $ln(1-x)$ becomes increasingly negative. \newline

      This also means that $-x -\frac{x^2}2$ will always be greater than or equal to $-x -\frac{x^2}2 -\frac{x^3}3 -\dotsi$ because of the fact that the two equations share their first two terms ($-x -\frac{x^2}2$), but the first equation has more negative terms. \newline
      \newline

      In the case that $x=0$, the two will be equal. In any other case where $x\in[0,1)$, equation 1 will be less than equation 2. Therefore, it can be said that $ln(1-x) \leq -x -\frac{x^2}2$.
    }
    %%%%%%%%%%%%%%
    \newpage
    
    \item {\bfseries (10 points)} For $x\in[0,1/2]$, prove that 
      $$\ln(1-x) \geq -x - \frac{x^2}{2\cdot 2^0} - \frac{x^2}{2\cdot 2^1} -\frac{x^2}{2\cdot 2^2} - \frac{x^2}{2\cdot 2^3} - \dotsi = -x-x^2.$$
  \newline
    %%% ANSWER %%%
    {\bfseries
      \newline
      \newline

      From part a, we know that, for $x\in[0,1/2]$ (which is a subset of $x\in[0,1)$), \newline
      $\ln(1-x) = -x - \frac{x^2}2 - \frac{x^3}3 - \dotsi.$ \newline
      \newline

      If we simplify the equation given in this part and compare it with the one from part a, we can see the following: \newline
      $ln(1-x)$ \hspace{39px} $\geq -x-\frac{x^2}{2*2^0}-\frac{x^2}{2*2^1}-\dotsi$ \newline
      $=$ \newline
      $-x-\frac{x^2}2-\frac{x^3}3-\dotsi \geq -x-\frac{x^2}2-\frac{x^2}4-\dotsi$ \newline

      For each equation, the first two terms are identical ($-x-\frac{x^2}2$). This means that the only terms to compare are the 3rd terms and beyond. If we set up an inequality using the 3rd term: \newline
      $-\frac{x^3}3 \geq -\frac{x^2}4$ \newline
      $\frac{x^3}3 \leq \frac{x^2}4$ \newline
      $\frac{x}3 \leq \frac14$ \newline
      $x \leq \frac34$ \newline
      \newline
      
      This means that $-x-\frac{x^2}2-\frac{x^3}3-\dotsi \geq -x-\frac{x^2}2-\frac{x^2}4-\dotsi$ for any $x \leq \frac34$. In this problem, $x\in[0,1/2]$, which is less than $3/4$. Therefore, we can conclude that 
      $\ln(1-x) \geq -x - \frac{x^2}{2\cdot 2^0} - \frac{x^2}{2\cdot 2^1} -\frac{x^2}{2\cdot 2^2} - \frac{x^2}{2\cdot 2^3} - \dotsi = -x-x^2.$
    }
    %%%%%%%%%%%%%%
  \newpage  
\end{enumerate}







%%%%%%%%%%%%%%%%%%%%%%%%%%%%%%%%%%%%%%%%%%%%%%%%%%
%%%%%%%%%%%% PROBLEM 2 %%%%%%%%%%%%%%%%%%%%%%%%%%%%
%%%%%%%%%%%%%%%%%%%%%%%%%%%%%%%%%%%%%%%%%%%%%%%%%%%
\item {\bfseries Tight Estimations.} 
  Provide meaningful upper-bounds and lower-bounds for the following expressions.
  \begin{enumerate}
  \item {\bfseries (10 points)} $S_n = \sum_{i=1}^n \ln i$,   \newline
    %%% ANSWER %%%
    {\bfseries
      \newline
      \newline
      This summation looks like the following: \newline
      $ln(1) + ln(2) + \dotsi + ln(n)$ \newline
      Which means that the function $f(x) = ln(x)$, an increasing function.\newline
      \newline

      For an increasing function $f(x)$, \newline
      $\int_0^x \! f(t) \, \mathrm{d}t. \leq f(x) \leq \int_1^{x+1} \! f(t) \, \mathrm{d}t$ \newline
      \newline
      Therefore, \newline
      $\int_0^x \! ln(t) \, \mathrm{d}t. \leq ln(x) \leq \int_1^{x+1} \! ln(t) \, \mathrm{d}t$ \newline
      \newline

      However, since the natural log of 0 is not possible, we increase the bounds of the lower-bound integral of $ln(x)$. To compensate, we add 1 to the resulting equation, giving our lower bound a possible error of up to 1. The resulting evaluation follows. \newline
      $1 + \int_1^x \! ln(t) \, \mathrm{d}t. \leq ln(x) \leq \int_1^{x+1} \! ln(t) \, \mathrm{d}t$ \newline
      Left Side:
      $1 + \int_1^x \! ln(t) \, \mathrm{d}t = 1 + (t*ln(t) - t)\Big|_x^1 = 2 + x*ln(x) - x$ \newline
      Right Side:
      $\int_1^{x+1} \! ln(t) \, \mathrm{d}t = (t*ln(t) - t)\Big|_1^{x+1} = (x+1)*ln(x+1) - x$ \newline
      \newline

      Therefore, the final bounds are: \newline
      $2 + x*ln(x) - x \leq \sum_{i=1}^n \ln i \leq (x+1)*ln(x+1) - x$
    }
    %%%%%%%%%%%%%%
   \newpage
  \item {\bfseries (10 points)} $A_n = n!$  \newline
    %%% ANSWER %%%
    {\bfseries
      \newline
      \newline
      $L_n \leq ln(n!) \leq U_n$ \newline
      $e^{L_n} \leq n! \leq e^{U_n}$ \newline
      \newline
      From part (a), we know that: \newline
      $2 + x*ln(x) - x \leq \sum_{i=1}^n \ln i \leq (x+1)*ln(x+1) - x$ \newline
      \newline

      Therefore, \newline
      $e^{2 + x*ln(x) - x} \leq n! \leq e^{(x+1)*ln(x+1) - x}$ \newline
    }
    %%%%%%%%%%%%%%
   \newpage
  \item {\bfseries (10 points)} $B_n =\binom{2n}{n} = \frac{(2n)!}{(n!)^2}$ \newline
    %%% ANSWER %%%
    {\bfseries
      \newline
      \newline
      When estimating fractions, we know that: \newline
      $\frac{L_{X_n}}{U_{Y_n}} \leq \frac{X_n}{Y_n} \leq \frac{U_{X_n}}{L_{Y_n}}$ \newline
      \newline

      From part (b), we know that: \newline
      $e^{2 + x*ln(x) - x} \leq n! \leq e^{(x+1)*ln(x+1) - x}$ \newline
      \newline

      So if $X_n = (2n)!$, \newline
      Using the information from part (b), it follows that: \newline
      $e^{2 + (2x)*ln(2x) - 2x} \leq (2n)! \leq e^{(2x+1)*ln(2x+1) - 2x}$ \newline
      \newline

      And if $Y_n = (n!)^2$, \newline
      Using the information from part (b), it follows that: \newline
      $(e^{2 + x*ln(x) - x})^2 \leq (n!)^2 \leq (e^{(x+1)*ln(x+1) - x})^2$ \newline
      \newline

      Therefore, \newline
      $\frac{e^{2 + (2x)*ln(2x) - 2x}}{(e^{(x+1)*ln(x+1) - x})^2}
      \leq
      \frac{(2n)!}{(n!)^2}
      \leq
      \frac{e^{(2x+1)*ln(2x+1) - 2x}}{(e^{2 + x*ln(x) - x})^2}
      $
    }
    %%%%%%%%%%%%%%
  \newpage  
  \end{enumerate} 
  


   
%%%%%%%%%%%%%%%%%%%%%%%%%%%%%%%%%%%%%%%%%%%%%%%%%%
%%%%%%%%%%%% PROBLEM 3 %%%%%%%%%%%%%%%%%%%%%%%%%%%%
%%%%%%%%%%%%%%%%%%%%%%%%%%%%%%%%%%%%%%%%%%%%%%%%%%%

\item {\bfseries Understanding Joint Distribution.}
  Recall that in the lectures we considered the joint distribution $(\T,\B)$ over the sample space $\{4,5,\dotsc,10\}\times\{\pred T,\pred F\}$, where $\T$ represents the time I wake up in the morning, and $\B$ represents whether I have breakfast or not. 
  The following table summarizes the joint probability distribution.%
  \begin{table}[h]
  \begin{center}\footnotesize 
  \begin{tabular}{|c|c|c|}\hline 
  $t$ & $b$ & \probX{\T=t,\B=b} \\\hline
  4 & \pred T & 0.01 \\\hline
  4 & \pred F & 0.05 \\\hline 
  5 & \pred T & 0 \\\hline
  5 & \pred F & 0.04 \\\hline 
  6 & \pred T & 0.1 \\\hline
  6 & \pred F & 0.20 \\\hline 
  7 & \pred T & 0.25 \\\hline
  7 & \pred F & 0.10 \\\hline 
  8 & \pred T & 0.10 \\\hline
  8 & \pred F & 0.05 \\\hline 
  9 & \pred T & 0.03 \\\hline
  9 & \pred F & 0.05 \\\hline 
  10 & \pred T & 0.01 \\\hline
  10 & \pred F & 0.01 \\\hline 
  \end{tabular} 
  \end{center}
  \end{table}
  
  Calculate the following probabilities.
  \begin{enumerate}
  \item {\bfseries (5 points)} Calculate the probability that I wake up at 8 a.m. or earlier, but do not have breakfast. 
    That is, calculate $\probX{\T\leq 8, \B= \pred F}$,   \newline
      %%% ANSWER %%%
      {\bfseries
        \newline
        $\probX{\T\leq 8, \B= \pred F} = \probX{\T= 4, \B=\pred F} + \probX{\T= 5, \B=\pred F} + \probX{\T= 6, \B=\pred F}+ \probX{\T= 7, \B=\pred F} + \probX{\T= 8, \B=\pred F}$
        \newline \newline
        $\probX{\T\leq 8, \B= \pred F} = 0.05 + 0.04 + 0.2 + 0.1 + 0.05 = 0.44$
      }
      %%%%%%%%%%%%%%
      \newpage

  \item {\bfseries (5 points)} Calculate the probability that I wake up at 8 a.m. or earlier. That is, calculate $\probX{\T\leq 8}$,    \newline
      %%% ANSWER %%%
      {\bfseries
        \newline
        $\probX{\T\leq 8} = \probX{\T= 4} + \probX{\T= 5} + \probX{\T= 6} + \probX{\T= 7} + \probX{\T= 8}$
        \newline
        $= (0.01 + 0.05) + (0 + 0.04) + (0.1 + 0.2) + (0.25 + 0.1) + (0.1 + 0.05)$
        \newline
        \newline
        $\probX{\T\leq 8}$ $= 0.9$
      }
      %%%%%%%%%%%%%%
  \vspace{0.5\textheight}

  \item {\bfseries (5 points)} Calculate the probability that I skip breakfast conditioned on the fact that I woke up at 8 a.m. or earlier. 
    That is, compute $\probX{\B=\pred F ~|~ \T\leq 8}$.   \newline
    %%% ANSWER %%%
    {\bfseries
      \newline
      By Bayes Rule, \newline
      $\probX{\B=\pred F ~|~ \T\leq 8} = \frac{\probX{\B=\pred F, \T\leq 8}}{\probX{\T\leq 8}}$ \newline \newline
      We know the numerator from part (a), and the denominator from part (b). Therefore, \newline
      $\probX{\B=\pred F ~|~ \T\leq 8} = \frac{\probX{\B=\pred F, \T\leq 8}}{\probX{\T\leq 8}} = \frac{0.44}{0.9} = 0.4888$
    }
    %%%%%%%%%%%%%%
  \newpage
  \end{enumerate} 
  
  \item {\bfseries Random Walk.}
  There is a frog sitting at the origin $(0,0)$ in the first quadrant of a two-dimensional Cartesian plane. The frog first jumps uniformly at random along the X-axis to some point $(\X,0)$, where $\X\in \{1,2,3,4,5,6\}$. 
  Then, it jumps uniformly at random along the Y-axis to some point $(\X,\Y)$, where $\Y \in \{1,2,3,4,5,6\}$. 
  So $(\X,\Y)$ represents the final position of the frog after these two jumps. 
  Note that $\X$ and $\Y$ are two independent random variables that are uniformly distributed over their respective sample spaces. 
  \begin{enumerate}
      \item {\bfseries (5 points)} What is the probability that the frog jumps more than 3 units along the Y-axis. That is, compute $\probX{\Y > 3}$. 
      \newline
      %%% ANSWER %%%
      {\bfseries
        \newline
        \newline
        There are 6 equally likely distances for the frog to jump on the Y-axis. Out of those 6, 3 distances are greater than 3 -- 4, 5, and 6. Therefore, \newline
        $\probX{\Y > 3} = 0.5$
      }
      %%%%%%%%%%%%%%
      \vspace{0.25\textheight}
      \item {\bfseries (10 points)} What is the probability that the final position of the frog is above the line $X+Y=7$. That is compute $\probX{\X+\Y >7}$?
      \newline
      %%% ANSWER %%%
      {\bfseries
        \newline
        There are 36 separate combinations of jumps. Out of those 36 combinations, there are 6 that satisfy $X+Y=7$: \newline
        1 and 6 \newline
        2 and 5 \newline
        3 and 4 \newline
        4 and 3 \newline
        5 and 2 \newline
        6 and 1 \newline
        \newline
        Therefore, \newline
        $\probX{\X+\Y >7} = \frac{6}{36} = \frac{1}{6} = 0.1666$
      }
      %%%%%%%%%%%%%%
      \vspace{0.25\textheight}
       \item {\bfseries (10 points)} What is the probability that the frog has jumped $2$ units along X-axis conditioned on the fact that its final position is above the line $X+Y=7$?
       That is, compute $\probX{\X=4 ~|~ \X+\Y >7}$?
      %%% ANSWER %%%
      \bfseries{
        \newline
        \newline
        By Bayes Rule: \newline
        $\probX{\X=4 ~|~ \X+\Y >7} = \frac{\probX{\X= 4, \X+\Y >7}}{\probX{\X+\Y >7}}$ \newline
        And by Chain Rule: \newline
        $\frac{\probX{\X= 4, \X+\Y >7}}{\probX{\X+\Y >7}} = \frac{\probX{\X= 4} * \probX{\X+\Y >7 ~|~ \X= 4}}{\probX{\X+\Y >7}}$ \newline
        \newline
        We know the denominator from part (b), and we can reduce $\probX{\X+\Y >7 ~|~ \X= 4}$ to $\probX{\Y >3}$, which we know from part (a). The final equation is: \newline
        $\frac{\probX{\X= 4} * \probX{\Y >3}}{\probX{\X+\Y >7}} = \frac{\frac{1}{6} * \frac{1}{2}}{\frac{1}{6}} = \frac{1}{2} = 0.5$
      }
      %%%%%%%%%%%%%%
  \end{enumerate}
  \newpage
  \item {\bfseries Coin Tossing Word Problem.}
  We have three (independent) coins represented by random variables $\C_1,\C_2,$ and $\C_3$.
  \begin{enumerate}[label=(\roman*)]
      \item The first coin has $\probX{\C_1=H}=\frac14,
      \probX{\C_2=T}=\frac34$, 
      \item The second coin has $\probX{\C_2=H}=\frac34$ and $\probX{\C_2=T}=\frac14$, and
      \item The third coin has $\probX{\C_3=H}=\frac14$ and $\probX{\C_3=T}=\frac34$. 
  \end{enumerate}
  
  Consider the following experiment. 
  \begin{enumerate}[label=(\Alph*)]
      \item Toss the first coin. Let the outcome of the first coin-toss be $\omega_1$. 
      \item If $\omega_1=H$, then we toss the second coin twice. 
        Otherwise, (i.e., if $\omega_1=T$) toss the third coin twice. 
        Let the two outcomes of this step be represented by $\omega_2$ and $\omega_3$. 
      \item Output $(\omega_1,\omega_2,\omega_3)$.
  \end{enumerate}
  Based on this experiment, compute the probabilities below. 

\begin{enumerate}
    \item {\bfseries (5 points)} 
  In the experiment mentioned above, what is the probability that a majority of the three outcomes $(\omega_1,\omega_2,\omega_3)$ are $H$ (head)?
  \newline
      %%% ANSWER %%%
      {\bfseries
        \newline
        There are four different possibilities where the majority of the three outcomes can be H: \newline
        (H H H), (H H T), (H T H), (T H H) \newline

        By the Chain Rule, \newline
        $\probX{\omega_1= H, \omega_2= H, \omega_3= H} = \probX{\omega_1= H} * \probX{\omega_2= H ~|~ \omega_1= H} * \probX{\omega_3= H ~|~ \omega_1= H, \omega_2= H}$ \newline
        $\probX{\omega_1= H, \omega_2= H, \omega_3= H}= \frac{1}{4} * \frac{3}{4} * \frac{3}{4} = \frac{9}{64} = 0.140625$ \newline

        $\probX{\omega_1= H, \omega_2= H, \omega_3= T} = \probX{\omega_1= H} * \probX{\omega_2= H ~|~ \omega_1= H} * \probX{\omega_3= T ~|~ \omega_1= H, \omega_2= H}$ \newline
        $\probX{\omega_1= H, \omega_2= H, \omega_3= T}= \frac{1}{4} * \frac{3}{4} * \frac{1}{4} = \frac{3}{64} = 0.046875$ \newline

        $\probX{\omega_1= H, \omega_2= T, \omega_3= H} = \probX{\omega_1= H} * \probX{\omega_2= T ~|~ \omega_1= H} * \probX{\omega_3= H ~|~ \omega_1= H, \omega_2= T}$ \newline
        $\probX{\omega_1= H, \omega_2= T, \omega_3= H}= \frac{1}{4} * \frac{1}{4} * \frac{3}{4} = \frac{3}{64} = 0.046875$ \newline

        $\probX{\omega_1= T, \omega_2= H, \omega_3= H} = \probX{\omega_1= T} * \probX{\omega_2= H ~|~ \omega_1= T} * \probX{\omega_3= H ~|~ \omega_1= T, \omega_2= H}$ \newline
        $\probX{\omega_1= T, \omega_2= H, \omega_3= H}= \frac{1}{4} * \frac{1}{4} * \frac{1}{4} = \frac{1}{64} = 0.015625$ \newline
        \newline
        Summing everything up, we get that the probability of a majority of the tree outcomes $(\omega_1, \omega_2, \omega_3)$ are $H$ (head) is: \newline
        $\frac{9}{64} + \frac{3}{64} + \frac{3}{64} + \frac{1}{64} = \frac{16}{64} = \frac{1}{4} = 0.25$
      }
      %%%%%%%%%%%%%%
      \newpage
      
    \item {\bfseries (5 points)} In the experiment mentioned above, what is the probability that a majority of the three outcomes are $H$, conditioned on the fact that the first outcome was $T$?
    \newline
      %%% ANSWER %%%
      {\bfseries
        \newline
        \newline
        There is only one case in which the majority of the three outcomes are $H$, conditioned on the fact that the first outcome was $T$: \newline
        $\probX{\omega_1= T, \omega_2= H, \omega_3= H}$ \newline
        \newline
        By Chain Rule, \newline
        $\probX{\omega_1= T, \omega_2= H, \omega_3= H} = \probX{\omega_1= T} * \probX{\omega_2= H ~|~ \omega_1= T} * \probX{\omega_3= H ~|~ \omega_1= T, \omega_2= H} = \frac{3}{4} * \frac{1}{4} * \frac{1}{4} = \frac{3}{64} = 0.046875$ \newline
      }
      %%%%%%%%%%%%%%
      \vspace{0.25\textheight}
      
    \item {\bfseries (5 points)} In the experiment mentioned above, what is the probability that a majority of the three outcomes are different from the first outcome?
   \newline
      %%% ANSWER %%%
      {\bfseries
        \newline
        \newline
        There are only two scenarios where a majority of the three outcomes are different from the first outcome, so the equation will be as follows: \newline
        $\probX{\omega_1= H, \omega_2= T, \omega_3= T} + \probX{\omega_1= T, \omega_2= H, \omega_3= H}$ \newline
        \newline

        By Chain Rule, \newline
        $\probX{\omega_1= H, \omega_2= T, \omega_3= T} + \probX{\omega_1= T, \omega_2= H, \omega_3= H} = \newline
        \probX{\omega_1= H} * \probX{\omega_2= T ~|~ \omega_1= H} * \probX{\omega_3= T ~|~ \omega_1= H, \omega_2= T} + \newline
        \probX{\omega_1= T} * \probX{\omega_2= H ~|~ \omega_1= T} * \probX{\omega_3= H ~|~ \omega_1= T, \omega_2= H} \newline
        = \frac{1}{4} * \frac{1}{4} * \frac{1}{4} + \frac{3}{4} * \frac{1}{4} * \frac{1}{4} = \frac{1}{64} + \frac{3}{64} = \frac{4}{64} = \frac{1}{16} = 0.0625
        $
      }
      %%%%%%%%%%%%%%
  
  
  \end{enumerate}


\end{enumerate}









\end{document}
