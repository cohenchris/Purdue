\documentclass[11pt]{article}
%% Packages
\usepackage[T1]{fontenc}
\usepackage{amssymb}
\usepackage{amsmath}
\usepackage{amsthm}
\usepackage{amsfonts}
\usepackage{mathtools}
\usepackage{xspace}
\usepackage{bm}
\usepackage{nicefrac}
\usepackage{commath}
\usepackage{bbm}
\usepackage{boxedminipage}
\usepackage{xparse}
\usepackage{xcolor}
\usepackage{float}
\usepackage{multirow}
\usepackage{graphicx}
\usepackage{caption}
\usepackage{subcaption}
\usepackage{ifthen}
\usepackage{algpseudocode}
\usepackage{proba} 
\usepackage{colortbl} 
%\usepackage{upgreek}
%\usepackage{paralist}
%\usepackage{enumitem}
%\usepackage{times}

\usepackage{tikz}
\usetikzlibrary{positioning}
\usetikzlibrary{fit}
\usetikzlibrary{calc}
\usetikzlibrary{backgrounds}
\usetikzlibrary{shapes}
\usetikzlibrary{patterns}
\usetikzlibrary{matrix}

\usepackage{geometry}

%\usepackage[pdftex,bookmarks=true,pdfstartview=FitH,colorlinks,linkcolor=blue,filecolor=blue,citecolor=blue,urlcolor=blue]{hyperref}
%    \urlstyle{sf}

%\usepackage{enumitem}


%% Document Design Choices
%\setlength{\parindent}{0pt}
%\setlength{\parskip}{10pt}
%\renewcommand{\labelitemi}{\ensuremath{\circ}}
%\geometry{left=1in, right=1in, top=1in, bottom=1in}


%% Marking
\newcommand\TODO[1]{{\color{red}{#1}}\xspace}
\newcommand\change[1]{{\color{red}\underline{#1}}\xspace}
\newcommand{\commentdg}[1]{{\color{blue}{#1}}\xspace}
\newcommand\commentop[1]{{\color{red}{#1}}\xspace}


%% Theorems and Environments
\newtheorem{conjecture}{Conjecture}
%\newtheorem{theorem}{Theorem}
%\newtheorem{importedtheorem}{Imported Theorem}
%\newtheorem{informaltheorem}{Informal Theorem}
%\newtheorem{definition}{Definition}
%\newtheorem{lemma}{Lemma}
\newtheorem{claim}{Claim}
%\newtheorem{corollary}[theorem]{Corollary}
%\newtheorem{observation}{Observation}
%\newtheorem{property}{Property}


\newenvironment{boxedalgo}
  {\begin{center}\begin{boxedminipage}{1\linewidth}}
  {\end{boxedminipage}\end{center}}

\newenvironment{inlinealgo}[1]
  {\noindent\hrulefill\\\noindent{\bf {#1}}}
  {\hrulefill}


%% References
\newcommand{\namedref}[2]{\hyperref[#2]{#1~\ref*{#2}}\xspace}
\newcommand{\lemmaref}[1]{\namedref{Lemma}{lem:#1}}
\newcommand{\theoremref}[1]{\namedref{Theorem}{thm:#1}}
\newcommand{\inftheoremref}[1]{\namedref{Informal Theorem}{infthm:#1}}
\newcommand{\claimref}[1]{\namedref{Claim}{clm:#1}}
\newcommand{\corollaryref}[1]{\namedref{Corollary}{corol:#1}}
\newcommand{\figureref}[1]{\namedref{Figure}{fig:#1}}
\newcommand{\equationref}[1]{\namedref{Equation}{eq:#1}}
\newcommand{\defref}[1]{\namedref{Definition}{def:#1}}
\newcommand{\observationref}[1]{\namedref{Observation}{obs:#1}}
\newcommand{\procedureref}[1]{\namedref{Procedure}{proc:#1}}
\newcommand{\importedtheoremref}[1]{\namedref{Imported Theorem}{impthm:#1}}
\newcommand{\informaltheoremref}[1]{\namedref{Informal Theorem}{infthm:#1}}
\newcommand{\sectionref}[1]{\namedref{Section}{sec:#1}}
\newcommand{\appendixref}[1]{\namedref{Appendix}{app:#1}}
\newcommand{\propertyref}[1]{\namedref{Property}{prop:#1}}


%% English
\newcommand{\ie}{\text{i.e.}\xspace}
\newcommand{\st}{\text{s.t.}\xspace}
\newcommand{\etal}{\text{et al.}\xspace}
\newcommand{\naive}{\text{na\"ive}\xspace}
\newcommand{\wrt}{\text{w.r.t.}\xspace}
\newcommand{\whp}{\text{w.h.p.}\xspace}
\newcommand{\resp}{\text{resp.}\xspace}
\newcommand{\eg}{\text{e.g.}\xspace}
\newcommand{\cf}{\text{cf.}\xspace}
\newcommand{\visavis}{\text{vis-\`a-vis}\xspace}

%% Authors
\newcommand{\damgard}{Damg{\aa}rd\xspace}


%% Crypto
\def\alice{{{Alice}}\xspace}
\def\bob{{{Bob}}\xspace}
\def\eve{{{Eve}}\xspace}
\newcommand{\poly}{\ensuremath{\mathrm{poly}}\xspace}
\newcommand{\polylog}{\ensuremath{\mathrm{polylog}\;}\xspace}
\newcommand{\negl}{\ensuremath{\mathsf{negl}}\xspace}
\newcommand{\secpar}{\ensuremath{{\lambda}}\xspace}
\newcommand{\zo}{\ensuremath{{\{0,1\}}}\xspace}

  %% Functionalities
  \newcommand{\func}[1]{\ensuremath{\cF_{\mathsf{#1}}}\xspace}
  \newcommand{\fcom}[0]{\func{com}}
  \newcommand{\fot}[0]{\func{ot}}
  
  %% Protocols
  % \newcommand{\prot}[1]{\ensuremath{\pi_{\mathsf{#1}}}\xspace}


%% Graph Theory
\newcommand{\ancest}[0]{\mathsf{Ancestors}}
\newcommand{\sibling}[0]{\mathsf{Siblings}}
\newcommand{\parent}[0]{\mathsf{parent}}
\newcommand{\leaves}[0]{\mathsf{leaves}}


%% Math Symbols
\renewcommand{\o}[0]{\ensuremath{\circ}\xspace}
\newcommand{\dist}[1]{\ensuremath{\left\langle{#1}\right\rangle}\xspace}
%\newcommand{\prob}[2]{\ensuremath{\dist{{#1}_1, \dotsc, {#1}_{#2}}}\xspace}
\newcommand{\ip}[2]{\ensuremath{\left\langle{#1},{#2}\right\rangle}\xspace}
\renewcommand{\vec}[1]{\ensuremath{\mathbf{#1}}\xspace}
\newcommand{\concat}[0]{\ensuremath{\circ}\xspace}
\newcommand{\nin}[0]{\ensuremath{\not\in}\xspace}
\newcommand{\xor}[0]{\ensuremath{\oplus}\xspace}
\newcommand{\rv}[1]{\ensuremath{\mathbf{#1}}\xspace}
\newcommand{\p}[1]{\ensuremath{^{{\left(#1\right)}}}\xspace}
\newcommand{\argmax}[0]{\ensuremath{\mathop{\mathrm{argmax}}~}\xspace}
\newcommand{\argmin}[0]{\ensuremath{\mathop{\mathrm{argmin}}~}\xspace}
\newcommand{\me}{\mathrm{e}}
\NewDocumentCommand\mathstack{>{\SplitList{;}}m}
  {\ensuremath{
    \begin{smallmatrix}
      \ProcessList{#1}{ \insertone }    
    \end{smallmatrix}
  }}
\newcommand{\insertone}[1]{\ensuremath{#1}\\}
\newcommand{\tuple}[1]{\ensuremath{{\left\langle{#1}\right\rangle}}\xspace}

%\newcommand{\argmax}{\operatornamewithlimits{argmax}}
%\DeclareMathOperator*{\argmax}{arg\,max}

  
  %% General
  \newcommand{\ceil}[1]{\ensuremath{\left\lceil{#1}\right\rceil}\xspace}
  \newcommand{\floor}[1]{\ensuremath{\left\lfloor{#1}\right\rfloor}\xspace}
  %\newcommand{\abs}[1]{\ensuremath{\left\vert{#1}\right\vert}\xspace}
  \newcommand{\lone}[1]{\ensuremath{\left\vert{#1}\right\vert}\xspace}
  \newcommand{\spnorm}[1]{\ensuremath{\left\Vert{#1}\right\Vert}\xspace}

  %% Renamed Symbols
  \newcommand{\eps}[0]{\varepsilon}
  \let\epsilon\eps
  %\let\phi\varphi
  
  %% Cal Alphabets
  \newcommand{\cA}{\ensuremath{{\mathcal A}}\xspace}
  \newcommand{\cB}{\ensuremath{{\mathcal B}}\xspace}
  \newcommand{\cC}{\ensuremath{{\mathcal C}}\xspace}
  \newcommand{\cD}{\ensuremath{{\mathcal D}}\xspace}
  \newcommand{\cE}{\ensuremath{{\mathcal E}}\xspace}
  \newcommand{\cF}{\ensuremath{{\mathcal F}}\xspace}
  \newcommand{\cG}{\ensuremath{{\mathcal G}}\xspace}
  \newcommand{\cH}{\ensuremath{{\mathcal H}}\xspace}
  \newcommand{\cI}{\ensuremath{{\mathcal I}}\xspace}
  \newcommand{\cK}{\ensuremath{{\mathcal K}}\xspace}
  \newcommand{\cL}{\ensuremath{{\mathcal L}}\xspace}
  \newcommand{\cM}{\ensuremath{{\mathcal M}}\xspace}
  \newcommand{\cN}{\ensuremath{{\mathcal N}}\xspace}
  \newcommand{\cO}{\ensuremath{{\mathcal O}}\xspace}
  \newcommand{\cP}{\ensuremath{{\mathcal P}}\xspace}
  \newcommand{\cQ}{\ensuremath{{\mathcal Q}}\xspace}
  \newcommand{\cR}{\ensuremath{{\mathcal R}}\xspace}
  \newcommand{\cS}{\ensuremath{{\mathcal S}}\xspace}
  \newcommand{\cT}{\ensuremath{{\mathcal T}}\xspace}
  \newcommand{\cU}{\ensuremath{{\mathcal U}}\xspace}
  \newcommand{\cV}{\ensuremath{{\mathcal V}}\xspace}
  \newcommand{\cW}{\ensuremath{{\mathcal W}}\xspace}
  \newcommand{\cX}{\ensuremath{{\mathcal X}}\xspace}
  \newcommand{\cY}{\ensuremath{{\mathcal Y}}\xspace}
  \newcommand{\cZ}{\ensuremath{{\mathcal Z}}\xspace}
  
  %% Bold Alphabets
  \newcommand{\bA}{\ensuremath{{\mathbf A}}\xspace}
  \newcommand{\bB}{\ensuremath{{\mathbf B}}\xspace}
  \newcommand{\bC}{\ensuremath{{\mathbf C}}\xspace}
  \newcommand{\bD}{\ensuremath{{\mathbf D}}\xspace}
  \newcommand{\bE}{\ensuremath{{\mathbf E}}\xspace}
  \newcommand{\bF}{\ensuremath{{\mathbf F}}\xspace}
  \newcommand{\bG}{\ensuremath{{\mathbf G}}\xspace}
  \newcommand{\bQ}{\ensuremath{{\mathbf Q}}\xspace}
  \newcommand{\bU}{\ensuremath{{\mathbf U}}\xspace}
  \newcommand{\bV}{\ensuremath{{\mathbf V}}\xspace}
  \newcommand{\bX}{\ensuremath{{\mathbf X}}\xspace}
  \newcommand{\bY}{\ensuremath{{\mathbf Y}}\xspace}
  \newcommand{\br}{\ensuremath{{\mathbf r}}\xspace}
  
  %% Black-board Bold Alphabets
  \newcommand{\bbA}{\ensuremath{{\mathbb A}}\xspace}
  \newcommand{\bbB}{\ensuremath{{\mathbb B}}\xspace}
  \newcommand{\bbC}{\ensuremath{{\mathbb C}}\xspace}
  \newcommand{\bbD}{\ensuremath{{\mathbb D}}\xspace}
  \newcommand{\bbE}{\ensuremath{{\mathbb E}}\xspace}
  \newcommand{\bbF}{\ensuremath{{\mathbb F}}\xspace}
  \newcommand{\bbG}{\ensuremath{{\mathbb G}}\xspace}
  \newcommand{\bbH}{\ensuremath{{\mathbb H}}\xspace}
  \newcommand{\bbI}{\ensuremath{{\mathbb I}}\xspace}
  \newcommand{\bbJ}{\ensuremath{{\mathbb J}}\xspace}
  \newcommand{\bbK}{\ensuremath{{\mathbb K}}\xspace}
  \newcommand{\bbN}{\ensuremath{{\mathbb N}}\xspace}
  \newcommand{\bbO}{\ensuremath{{\mathbb O}}\xspace}
  \newcommand{\bbP}{\ensuremath{{\mathbb P}}\xspace}
  \newcommand{\bbQ}{\ensuremath{{\mathbb Q}}\xspace}
  \newcommand{\bbR}{\ensuremath{{\mathbb R}}\xspace}
  \newcommand{\bbS}{\ensuremath{{\mathbb S}}\xspace}
  \newcommand{\bbT}{\ensuremath{{\mathbb T}}\xspace}
  \newcommand{\bbU}{\ensuremath{{\mathbb U}}\xspace}
  \newcommand{\bbV}{\ensuremath{{\mathbb V}}\xspace}
  \newcommand{\bbZ}{\ensuremath{{\mathbb Z}}\xspace}
  
  %% Fraktur Alphabets
  \newcommand{\fP}{\ensuremath{{\mathfrak P}}\xspace}
  \newcommand{\fR}{\ensuremath{{\mathfrak R}}\xspace}
  \newcommand{\fX}{\ensuremath{{\mathfrak X}}\xspace}
  
  %% Overline Aphabets
  \newcommand{\op}{\ensuremath{{\overline p}}\xspace}
  
  %% Hat Aphabets
  \newcommand{\hf}{\ensuremath{{\widehat f}}\xspace}
  \newcommand{\hg}{\ensuremath{{\widehat g}}\xspace}
  
  %% Fractions
  \newcommand{\half}{\ensuremath{\frac12}\xspace}
  
  %% Set
  \newcommand{\comp}[1]{\ensuremath{\overline{{#1}}}\xspace}
  
  %% Models
  \newcommand{\defeq}[0]{\ensuremath{{\;\vcentcolon=\;}}\xspace}
  \newcommand{\eqdef}[0]{\ensuremath{{\;=\vcentcolon\;}}\xspace}
  \newcommand{\entails}[0]{\ensuremath{{\;\models\;}}\xspace}
  
  %% Matrix
  \newcommand{\tran}[0]{\ensuremath{^{\mathsf{T}}}\xspace}
  
  %% Probability and Distributions
  \newcommand{\event}[1]{\ensuremath{\mathsf{#1}}\xspace}
  \newcommand{\supp}[0]{\ensuremath{\mathsf{Supp}}\xspace}
  \let\pr\prob
  %\newcommand{\pr}[0]{\mathop{\mathrm{Pr}}\xspace}
  \newcommand{\var}[0]{\mathop{\mathrm{Var}}\xspace}
  \newcommand{\cov}[0]{\mathop{\mathrm{Cov}}\xspace}
  %\renewcommand{\Pr}[0]{\mathrm{\generateerror}\xspace}
  %\newcommand{\E}[0]{\mathop{\bbE}\xspace}
  \newcommand{\getsr}[0]{\mathbin{\stackrel{\mbox{\,\tiny \$}}{\gets}}}
  \newcommand{\drawn}{\ensuremath{\sim}\xspace}
  %\newcommand{\sd}[2]{\ensuremath{\mathbf{\Delta}\left({#1},{#2}\right)}\xspace}
  \newcommand{\sd}[2]{\ensuremath{\mathrm{SD}\left({#1},{#2}\right)}\xspace}
  \newcommand{\hd}{\ensuremath{\mathrm{HD}}\xspace}
  \newcommand{\kl}[2]{\ensuremath{\mathrm{D_{KL}}\left({#1},{#2}\right)}\xspace}
  \newcommand{\iid}[0]{\text{i.i.d.}\xspace}
  \newcommand{\ent}[0]{\ensuremath{\mathrm{H}}\xspace}
  
    %% Random Variables
    \newcommand{\rvA}{\rv{A}}
    \newcommand{\rvB}{\rv{B}}
    \newcommand{\rvC}{\rv{C}}
    \newcommand{\rvD}{\rv{D}}
    \newcommand{\rvG}{\rv{G}}
    \newcommand{\rvP}{\rv{P}}
    \newcommand{\rvQ}{\rv{Q}}
    \newcommand{\rvR}{\rv{R}}
    \newcommand{\rvS}{\rv{S}}
    \newcommand{\rvT}{\rv{T}}
    \newcommand{\rvU}{\rv{U}}
    \newcommand{\rvV}{\rv{V}}
    \newcommand{\rvW}{\rv{W}}
    \newcommand{\rvX}{\rv{X}}
    \newcommand{\rvY}{\rv{Y}}
    \newcommand{\rvZ}{\rv{Z}}
    
    \newcommand{\rvm}{\rv{m}}
    \newcommand{\rvr}{\rv{r}}
    \newcommand{\rvx}{\rv{x}}
    \newcommand{\rvy}{\rv{y}}
    \newcommand{\rvc}{\rv{c}}

  %% Binary Operators
  \newcommand{\band}[0]{\ensuremath{~\wedge~}\xspace}
  \newcommand{\bor}[0]{\ensuremath{~\vee~}\xspace}
  \let\leq\leqslant
  \let\le\leqslant
  \let\geq\geqslant
  \let\ge\geqslant
  
  %% Combinatorial
  \renewcommand\choose[2]{\ensuremath{{
    \left(\begin{matrix}
      {#1}\\
      {#2}
    \end{matrix}\right)}}\xspace}
  \newcommand\smallchoose[2]{\ensuremath{{
    \left(\begin{smallmatrix}
      {#1}\\
      {#2}
    \end{smallmatrix}\right)}}\xspace}
  \newcommand{\fallfact}[2]{\ensuremath{{#1}_{\left({#2}\right)}}\xspace}
  %\newcommand{\fact}[1]{\ensuremath{{#1}!}\xspace}
  
  %% Set Operations
  \newcommand{\union}[0]{\ensuremath{\cup}\xspace}
  \newcommand{\intersect}[0]{\ensuremath{\cap}\xspace}
  \newcommand{\setdiff}[0]{\ensuremath{\Delta}\xspace}
  

%% Algorithms, Predicates
\newcommand{\pred}[1]{\ensuremath{\mathsf{#1}}\xspace}


%% Local Terms
\newcommand{\mycite}[1]{{\color{brown}{{#1}}}\xspace}
\newcommand{\wt}[0]{\ensuremath{\mathsf{wt}}\xspace}
\newcommand{\Inf}[0]{\ensuremath{\pred{Inf}}\xspace}
%\newcommand{\Var}[0]{\ensuremath{\pred{Var}}\xspace}
\newcommand{\mac}[0]{\pred{Mac}} 
\newcommand{\sk}[0]{\pred{sk}} 
\newcommand{\pk}[0]{\pred{pk}} 
\newcommand{\gen}[0]{\pred{Gen}} 
\newcommand{\enc}[0]{\pred{Enc}} 
\newcommand{\dec}[0]{\pred{Dec}} 
\newcommand{\SK}[0]{\pred{sk}} 
\newcommand{\capprox}[0]{\approx\p c}
\renewcommand{\tag}[0]{\pred{Tag}} 
\newcommand{\sign}[0]{\pred{Sign}} 
\newcommand{\ver}[0]{\pred{Ver}} 
\newcommand{\simu}[0]{\pred{Sim}} 
\newcommand{\ball}[0]{\pred{Ball}\xspace}
\newcommand{\vol}[0]{\pred{Vol}\xspace}


\newcommand{\pout}[0]{\ensuremath{\p{\text{out}}}\xspace}
\newcommand{\pin}[0]{\ensuremath{\p{\text{in}}}\xspace}

\newcommand{\Ext}[0]{\ensuremath{\pred{Ext}}\xspace}
\newcommand{\inv}[0]{\pred{inv}}



\usepackage{fancyhdr}   
\pagestyle{fancy}      
\lhead{CS 355, FALL 2020}               
\rhead{Name: Christopher Cohen}

\usepackage[strict]{changepage}  
\newcommand{\nextoddpage}{\checkoddpage\ifoddpage{\ \newpage\ \newpage}\else{\ \newpage}\fi}  


\begin{document}

\title{Homework 2}

\date{}

\maketitle 

\thispagestyle{fancy}  
\pagestyle{fancy}      


% 21 QUESTIONS

\begin{enumerate}
%%%%%%%%%%%%%%%%%%%%%%%%%%%%%%%%%%%%%%%%%%%%%%%%%%
%%%%%%%%%%%% PROBLEM 1 %%%%%%%%%%%%%%%%%%%%%%%%%%%%
%%%%%%%%%%%%%%%%%%%%%%%%%%%%%%%%%%%%%%%%%%%%%%%%%%%
\item {\bfseries Some properties of $(\bbZ_p^*,\times)$ (25 points).} 
  Recall that $\bbZ_p^*$ is the set $\{1,\dotsc,p-1\}$ and $\times$ is integer multiplication $\mod p$, where $p$ is a prime. 
  For example, if $p=5$, then $2\times 3$ is $1$. 
  In this problem we shall prove that $(\bbZ_p^*,\times)$ is a group, when $p$ is any prime. 
  The only part missing in the lecture was the proof that every $x\in\bbZ_p^*$ has an inverse. 
  We will find the inverse of any element $x\in\bbZ_p^*$. 
  \begin{enumerate}
  \item(10 points) Recall $\binom pk \defeq \frac{p!}{k!(p-k)!}$. 
    For a prime $p$, prove that $p$ divides $\binom pk$, if $k\in\{1,2,\dotsc,p-1\}$. \newline
%    \TODO{$p$ divides $\binom  {p^{r}}{k}$ for $k\in \{1,2,\dots,p^r-1\}$, maybe concrete $r$ like $r=2$ suffices}\newline
  %%%%% ANSWER %%%%%
  {\bfseries
      \newline
      \newline
      $\frac{p!}{k!(p-k)!}$ can be simplified to $\frac{p(p-1)(p-2)\dotsi(p-k+1)}{k!}$. \newline
      \newline

      This is because $p$ is always going to be bigger than $k$, so when all of the factorials are expanded, some of the lower terms of $p!$ will be cancelled out by the terms of $(p-k)!$. For example, if $p$ is 5 and $k$ is 2, $p!$ will expand to $5*4*3*2*1$, and $(p-k)!$ will expand to $3*2*1$. Since $p!$ is in the numerator, and $(p-k)!$ is in the denominator, there will only be a $5*4$ left in the numerator after simplification. \newline
      \newline

      We know that $p$ will always be present in the numerator because it cannot be divided by any $k$ and is always greater than $k$. Therefore, since $p$ is still present in the numerator, we can conclude that for a prime $p$, $p$ divides $\binom pk$ if $k\in\{1,2,\dotsc,p-1\}$.
  }
  %%%%%%%%%%%%%%%%%%
   \newpage
  \item(10 points) Recall that $(1+x)^p = \sum_{k=0}^p \binom pk x^k$. 
    Prove by induction on $x$ that, for any $x\in\bbZ_p^*$, we have  
      $$\overbrace{x\times x \times \dotsi \times x}^{p\text{-times}}=x$$\newline
  %%%%% ANSWER %%%%%
  {\bfseries
      Essentially, we must prove that $x^p = x\mod p$. \newline

      For proof by induction, we start with a base case. In this case, it will be when $x=1$, since that's the lowest possible value in the group $\bbZ_p^*$. For $x=1$, we can clearly see that $1^p = 1\mod p$. Therefore, the base case holds true. \newline

      The next step is to assume that, for an arbitrary value $y\in \bbZ_p^*$,

      $y^p= y\mod p$.

      Now, we must prove that it works for a value $(1+y)\in \bbZ_p^*$. In other words, we are proving the statement that $(1+y)^p = (1+y)\mod p$. \newline


      We can start by evaluating what was given to us, $(1+y)^p = \sum_{k=0}^p \binom pk y^k$. This expands into the following: \newline
      $ = \binom p0 y^0 + \binom p1 y^1 + \dotsi + \binom{p}{p-1} y^{p-1} + \binom pp y^p$ \newline
      Since $\binom p0 y^0 = 1$ and $\binom pp y^p = y^p$, this further simplifies into: \newline
      $=1 + \binom p1 y^1 + \dotsi + \binom{p}{p-1} y^{p-1} + y^p$ \newline

      We also proved in part (a) that, for any $k\in \{1,2,\dotsi,p-1\}$, $p$ divides $\binom pk$. Therefore, if we take the above equation$\mod p$, then all of the terms with coefficients between $\binom p1$ to $\binom p{p-1}$, inclusive, will cancel out, leaving us with the equation: \newline

      $=(1 + \binom p1 y^1 + \dotsi + \binom{p}{p-1} y^{p-1} + y^p) \mod p$ \newline
      $=(1+y^p) \mod p$ \newline

      Since we assume that $y^p\mod p = y$, and we know that $1\mod p = 1$, we can further simplify the right side, giving us the final equation: \newline
      $=(1+y)\mod p$

      And when joined with the equation in the problem statement, \newline
      $(1+y)^p = (1+y)\mod p$

      which is the equation that we desired. \newline

      Therefore, by induction, $x^p\mod p = x$.
  }
  %%%%%%%%%%%%%%%%%%
   \newpage
   \item(5 points) For $x\in\bbZ_p^*$, prove that the inverse of $x\in\bbZ_p^*$ is given by 
     $$\overbrace{x\times x \times \dotsi \times x}^{(p-2)\text{-times}}$$ 
  That is, prove that $x^{p-1}=1\mod p$, for any prime $p$ and $x\in\bbZ_p^*$.      
     \newline
  %%%%% ANSWER %%%%%
  {\bfseries
      \newline
      \newline
      From the previous part, we proved that $x^p= x\mod p$.

      Simplifying, if we divide both sides by $x$, we get the equation: \newline
      $x^{p-1}=1\mod p$ \newline
      Which is the equation that we are looking for. \newline

      Therefore, the inverse of $x\in \bbZ_p^*$ is given by $x^{p-1} = 1\mod p$ for any prime $p$ and any $x\in \bbZ_p^*$.
  }
  %%%%%%%%%%%%%%%%%%
   \newpage
  \end{enumerate} 
  

  


  
%%%%%%%%%%%%%%%%%%%%%%%%%%%%%%%%%%%%%%%%%%%%%%%%%%
%%%%%%%%%%%% PROBLEM 2 %%%%%%%%%%%%%%%%%%%%%%%%%%%%
%%%%%%%%%%%%%%%%%%%%%%%%%%%%%%%%%%%%%%%%%%%%%%%%%%%

 
 \item    {\bfseries Understanding Groups: Part one (30 points).} 
    Recall that when we defined a group $(G,\circ)$, we stated that there exists an element $e$ such that for all $x\in G$ we have $x\circ e=x$. 
  Note that $e$ is ``applied on $x$ from the right.'' 
  
  Similarly, for every $x\in G$, we are guaranteed that there exists $\inv(x)\in G$ such that $x\circ\inv(x)=e$. 
  Note that $\inv(x)$ is again ``applied to $x$ from the right.'' 
  
  In this problem, however, we shall explore the following questions: (a) Is there an ``identity from the left?,'' and (b) Is there an ``inverse from the left?'' 
  
  We shall formalize and prove these results in this question. 
  \begin{enumerate}
   \item (5 points) Prove that it is impossible that there exists $a,b,c\in G$ such that $a\neq b$ but $a\circ c = b\circ c$. \newline
  %%%%% ANSWER %%%%%
  {\bfseries
      \newline
      \newline
      Looking at the equation $a\circ c = b\circ c$, we can $\circ$ both sides of the equation by $inv(c)$.  Since we also know from the problem statement that $x\circ inv(x) = e$, the following can be deduced:

      $a\circ c = b\circ c \newline
      a\circ (c\circ inv(c)) = b\circ (c\circ inv(c)) \newline
      a\circ e = b\circ e \newline
      a = b$ \newline

      $a$ has to equal $b$, so we know that it is impossible that there exists $a,b,c\in G$ such that $a \neq b$ but $a\circ c = b\circ c$.
  }
  %%%%%%%%%%%%%%%%%%
    \newpage
  \item (6 points) Prove that $e\circ x = x$, for all $x\in G$.  \newline
  %%%%% ANSWER %%%%%
  {\bfseries
      \newline
      \newline
      Due to the identity property of a group, we know that $\exists e\in G$ such that for all $x \in G$, $x\circ e = x$. \newline

      Also, due to the inverse property of a group, we know that for every element $x\in G$, $\exists inv(x)\in G$ such that $x\circ inv(x) = e$. \newline

      Therefore, using associativity, we can deduce the following: \newline

      $e\circ x = x \newline
      (x\circ inv(x))\circ x = x \newline
      x\circ (inv(x)\circ x) = x \newline
      x\circ e = x$ \newline

      Since $x\circ e = x$ is 100\% true (it's a property of a group), we can say with confidence that $e\circ x = x$.

  }
  %%%%%%%%%%%%%%%%%%
     \newpage
  \item (6 points) Prove that if there exists an element $\alpha \in G$ such that for \textbf{some} $x\in G$, we have $\alpha \circ x=x$, then $\alpha=e$.  \newline
  (Remark: Note that these two steps prove that the ``left identity'' is identical to the right identity $e$.) 
  \newline 
  %%%%% ANSWER %%%%%
  {\bfseries
      \newline
      \newline
      From part (a), we know that it is impossible that $a\circ b = a\circ c$ where $b\neq c$. \newline


      This means that there is only ONE element $x$ where $a\circ x = x$. \newline

      We also know that, by the identity property of a group, $\exists e\in G$ such that for all $a\in G$, $a\circ e = a$. \newline

      Therefore, since we know that there is only ONE element that holds for $a\circ x = x$, and applying the identity element to a variable doesn't change it (which is what is happening here), we can say with confidence that $a$ must equal $e$.


  }
  %%%%%%%%%%%%%%%%%%
     \newpage
  
  
  
  \item (8 points) Prove that $\inv(x)\circ x = e$.  \newline
  %%%%% ANSWER %%%%%
  {\bfseries
      \newline
      \newline
      Say that an element $x$ has an inverse from the left, $i_L$, and an inverse from the right, $i_R$. This means that:

      $i_L\circ x = e$ and $x\circ i_R = e$ \newline

      Consider the equation $i_L\circ x\circ i_R$. Using the associativity property of a group, and the fact that a group has both a left and right identity (as proved in part (c)), we know that: \newline

      $(i_L\circ x)\circ i_R = i_L\circ (x\circ i_R) \newline
      e\circ i_R = i_L\circ e \newline
      i_R = i_L$ \newline

      Since the inverse from the left $i_L$ equals the inverse from the right $i_R$, and the problem statement says that there exists $inv(x)\in G$ such that $x\circ inv(x) = e$, we know for a fact that $inv(x)\circ x = e$.
  }
  %%%%%%%%%%%%%%%%%%
     \newpage
  \item (5 points) Prove that if there exists an element $\alpha\in G$ and $x\in G$ such that $\alpha\circ x=e$, then $\alpha=\inv(x)$.  \newline
  (Remark: Note that these two steps prove that the ``left inverse of $x$'' is identical to the right inverse $\inv(x)$. ) \newline
  %%%%% ANSWER %%%%%
  {\bfseries
      \newline
      \newline
      In part (d), we already proved that $inv(x)\circ x = x\circ inv(x) = e$. \newline
      The important equation here is that $inv(x)\circ x = e$. \newline
 
      So, since we also know from part (a) that there can only be ONE variable where $a\circ x = e$, we can confidently say that $a$ MUST equal $inv(x)$.
  }
  %%%%%%%%%%%%%%%%%%
     \newpage
 
 \end{enumerate}
 
    
  
%%%%%%%%%%%%%%%%%%%%%%%%%%%%%%%%%%%%%%%%%%%%%%%%%%
%%%%%%%%%%%% PROBLEM 3 %%%%%%%%%%%%%%%%%%%%%%%%%%%%
%%%%%%%%%%%%%%%%%%%%%%%%%%%%%%%%%%%%%%%%%%%%%%%%%%%

 
 \item {\bfseries Understanding Groups: Part Two (15 points).}
  In this part, we will prove a crucial property of inverses in groups -- they are unique. And finally, using this property, we will prove a result that is crucial to the proof of security of one-time pad over the group $(G,\circ)$. 
 
 \begin{enumerate} 
   \item(9 points) Suppose $a,b\in G$. 
     Let $\inv(a)$ and $\inv(b)$ be the inverses of $a$ and $b$, respectively (\ie, $a\circ\inv(a)=e$ and $b\circ\inv(b)=e$). 
     Prove that $\inv(a)=\inv(b)$ if and only if $a=b$.  \newline
  %%%%% ANSWER %%%%%
  {\bfseries
      \newline
      \newline
      Say we have an inverse $i\in G$, for 2 separate elements $a,b \in G$, where $a\circ i = e$ and $b\circ i = e$. Due to the fact that there is a left and right inverse AND identity (proved in question 2), we can derive the following: \newline

      $a = a\circ e \newline
      = a\circ (b\circ i) \newline
      = a\circ (i\circ b) \newline
      = (a\circ i)\circ b \newline
      = e\circ b \newline
      = b$ \newline
      $\therefore a=b$ \newline

      Since we simplified to $a=b$, we know that $inv(a)=inv(b)$ if and only if $a=b$.
  }
  %%%%%%%%%%%%%%%%%%
    % \end{itemize}
     \newpage
  
  \item (6 points) Suppose $m\in G $ is a message and $c\in G$ is a cipher text. 
    Prove that there exists a unique $\sk\in G$ such that $m\circ \sk = c$.  \newline
  %%%%% ANSWER %%%%%
  {\bfseries
      \newline
      \newline
      Say we have $sk_1\in G$ where $m\circ sk_1 = c$. \newline
      And another $sk_2\in G$ where $m\circ sk_2 = c$. \newline
      \newline

      We know from the previous question that $inv(a)=inv(b)$ if and only if $a=b$. \newline
      We also know that decryption for one-time pad with secret key $sk\in G$ is defined as $c\circ inv(sk) = m$. \newline

      Using all of the above equations, we can do the following: \newline

      $c \circ inv(sk_1) = m \newline
      = (m\circ sk_2) \circ inv(sk_1) = m \newline
      = m \circ (sk_2 \circ inv(sk_1) = m$ \newline
      
      In problem 2, we have already proved that there can only be one element that holds for an equation $m\circ x = m$ - the identity element $e$. This means that $\sk_2\circ inv(sk_1) = e$. We also proved in part 2 that an element has a unique inverse. Therefore, the only way that $sk_2\circ inv(sk_1) = e$ is if $sk_1=sk_2$.
      Given all of this information, we can conclude that there is a unique secret key $sk$ such that $m\circ sk = c$.
  }
  %%%%%%%%%%%%%%%%%%
     \newpage
   \newpage
  \end{enumerate}
  
 \newpage 


%%%%%%%%%%%%%%%%%%%%%%%%%%%%%%%%%%%%%%%%%%%%%%%%%%
%%%%%%%%%%%% PROBLEM 4 %%%%%%%%%%%%%%%%%%%%%%%%%%%%
%%%%%%%%%%%%%%%%%%%%%%%%%%%%%%%%%%%%%%%%%%%%%%%%%%%

\item {\bfseries Calculating Large Powers mod $p$ (15 points).} 
  Recall that we learned the repeated squaring algorithm in class. 
\newline 
  Calculate the following using this concept \newline
  $$ 11^{2020^{2020}+2020} \pmod{101} $$
  (Hint: Note that $101$ is a prime number and before applying repeated squaring algorithm try to simplify the problem using what you learned in part C of question 1). \newline 
  %%%%% ANSWER %%%%%
  {\bfseries
      \newline
      \newline
      $11^{2020^{2020}+2020} \pmod{101}$ can be expanded to $11^{2020^{2020}} * 11^{2020} \pmod{101}$. \newline
      In part C of question 1, we derived the equation $x^{p-1}=1\mod p$, for any prime number $p$. We know that $101$ is a prime number, so with $x=11$, we have the equation $11^{100}=1\mod 101$. \newline

      Focusing on the lone $11^{2020}$ (note that the$\mod 101$ is abstracted out for readability): \newline
      $11^{2020} \newline
      =11^{100*20+20} \newline
      =(11^{100})^{20}*11^{20} \newline
      =1^{20}*11^{20} \newline
      =11^{20}$ \newline
      Doing the same for $11^{2020^{2020}}$, and that $11^{2020} = 11^{20}$ ($\mod 101$ is implicit): \newline
      $11^{2020^{2020}}\newline
      =11^{20^{2020}} \newline
      =11^{20^{100*20+20}} \newline
      =((11^{20})^{^{100}})^{^{20}}*11^{20^{20}} \newline
      =1^{20}*11^{20^{20}} \newline
      =11^{20^{20}} \newline
      =(11^{20^2})^{^{10}} \newline
      =(11^{400})^{^{10}} \newline
      =((11^{100})^{4})^{^{10}} \newline
      =(1^4)^{^{10}} \newline
      =1$

      Therefore, the equation from the 1st paragraph simplifies into: \newline
      $11^{2020^{2020}}*11^{2020} \pmod{101} \newline
      =1*11^{20} \pmod{101} \newline
      =11^{20} \mod 101$ \newline

      The repeating squares algorithm for this problem defines the following: \newline
      $\alpha_0 = 11^{2^0} \mod 101 = 11^1 \mod 101 = 11 \mod 101 = 11 \newline
      \alpha_1 = 11^{2^1} \mod 101 = \alpha_0 * \alpha_0 \pmod {101} = 121 \mod 101 = 20 \newline
      \alpha_2 = 11^{2^2} \mod 101 = \alpha_1 * \alpha_1 \pmod {101} = 400 \mod 101 = 97 \newline
      \alpha_3 = 11^{2^3} \mod 101 = \alpha_2 * \alpha_2 \pmod {101} = 9409 \mod 101 = 16 \newline
      \alpha_4 = 11^{2^4} \mod 101 = \alpha_3 * \alpha_3 \pmod {101} = 256 \mod 101 = 54 \newline$

      Our simplified equation can be broken apart into: \newline
      $11^{20} \mod 101\newline
      =11^{16+4} \mod 101 \newline
      =11^{16} * 11^4 \pmod{101}\newline
      =\alpha_4 * \alpha_2 \pmod{101} \newline
      =54*97 \pmod{101} \newline
      =5238 \mod 101 \newline
      =87$
      Therefore, $11^{2020^{2020}+2020} = 87$

  }
  %%%%%%%%%%%%%%%%%%
     \newpage
  
  
 
   
%%%%%%%%%%%%%%%%%%%%%%%%%%%%%%%%%%%%%%%%%%%%%%%%%%
%%%%%%%%%%%% PROBLEM 5 %%%%%%%%%%%%%%%%%%%%%%%%%%%%
%%%%%%%%%%%%%%%%%%%%%%%%%%%%%%%%%%%%%%%%%%%%%%%%%%%


\item {\bfseries Practice with Fields (20 points).} 
  We shall work over the field $(\bbZ_5,+,\times)$. 
  
  \begin{enumerate}
  \item (5 points) Addition Table. 
    The $(i,j)$-th entry in the table is $i+j$. 
    Complete this table. 
    You do not need to fill the black cells because the addition is commutative. 
  
    \begin{table}[H]
    \begin{center}
    \begin{tabular}{|c|c|c|c|c|c|}\hline
     & 0 & 1 & 2 & 3 & 4 \\\hline 
    0 & 0 & 1 & 2 & 3 & 4 \\\hline
    1 & \cellcolor{black} & 2 & 3 & 4 & 0 \\\hline
    2 & \cellcolor{black} & \cellcolor{black} & 4 & 0 & 1 \\\hline
    3 & \cellcolor{black} & \cellcolor{black} & \cellcolor{black} & 1 & 2 \\\hline
    4 & \cellcolor{black} & \cellcolor{black} & \cellcolor{black} & \cellcolor{black} & 3 \\\hline
    \end{tabular}
    \end{center}
    \caption{Addition Table.}
    \end{table}
  
  \item (5 points) Multiplication Table. 
    The $(i,j)$-th entry in the table is $i\times j$. 
    Complete this table. 
  
    \begin{table}[H]
    \begin{center}
    \begin{tabular}{|c|c|c|c|c|c|}\hline
     & 0 & 1 & 2 & 3 & 4 \\\hline 
    0 & 0 & 0 & 0 & 0 & 0 \\\hline
    1 & \cellcolor{black} & 1 & 2 & 3 & 4 \\\hline
    2 & \cellcolor{black} & \cellcolor{black} & 4 & 1 & 3 \\\hline
    3 & \cellcolor{black} & \cellcolor{black} & \cellcolor{black} & 4 & 2 \\\hline
    4 & \cellcolor{black} & \cellcolor{black} & \cellcolor{black} & \cellcolor{black} & 1 \\\hline
    \end{tabular}
    \end{center}
    \caption{Multiplication Table.}
    \end{table}
    
  \item (5 points) Additive and Multiplicative Inverses. 
    Write the additive and multiplicative inverses in the table below. 
  
    \begin{table}[H]
    \begin{center}
    \begin{tabular}{|l|c|c|c|c|c|}\hline
    & 0 & 1 & 2 & 3 & 4 \\\hline 
    Additive Inverse & 0 & 4 & 3 & 2 & 1 \\\hline
    Multiplicative Inverse & \cellcolor{black} & 1 & 3 & 2 & 4 \\\hline 
    \end{tabular}
    \end{center}
    \caption{Additive and Multiplicative Inverses Table.}
    \end{table} 
  
  \item (5 points) Division Table. 
    The $(i,j)$-th entry in the table is $i/ j$. 
    Complete this table. 
  
    \begin{table}[H]
    \begin{center}
    \begin{tabular}{|c|c|c|c|c|}\hline
      & 1 & 2 & 3 & 4 \\\hline 
    0 & 0 & 0 & 0 & 0 \\\hline
    1 & 1 & 3 & 2 & 4 \\\hline
    2 & 2 & 1 & 4 & 3 \\\hline
    3 & 3 & 4 & 1 & 2 \\\hline
    4 & 4 & 2 & 3 & 1 \\\hline
    \end{tabular}
    \end{center}
    \caption{Division Table.}
    \end{table}
  
  \end{enumerate} 




%%%%%%%%%%%%%%%%%%%%%%%%%%%%%%%%%%%%%%%%%%%%%%%%%%
%%%%%%%%%%%% PROBLEM 6 %%%%%%%%%%%%%%%%%%%%%%%%%%%%
%%%%%%%%%%%%%%%%%%%%%%%%%%%%%%%%%%%%%%%%%%%%%%%%%%%
\newpage
\item {\bfseries Order of an Element in $(\bbZ_p^*,\times)$. (20 points)} The \textit{order} of an element $x$ in the multiplicative group $(\bbZ_p^*,\times)$ is the smallest positive integer $h$ such that $x^h = 1 \mod p $. For example, the order of 2 in $(\bbZ_5^*,\times)$ is 4, and the order of $4$ in $(\bbZ_5^*,\times)$ is 2. 
\begin{enumerate}
    \item (5 points) What is the order of 5 in $(\bbZ_{11}^*,\times)$? \newline
  %%%%% ANSWER %%%%%
  {\bfseries
      \newline
      \newline
      $5^1\mod111 = 5\mod 11 = 5 \newline
      5^2\mod 111 = 25\mod 11 = 3 \newline
      5^3\mod 111 = 125\mod 11 = 4 \newline
      5^4\mod 111 = 625\mod 11 = 9 \newline
      5^5\mod 111 = 3125\mod 11 = 1$ \newline

      Therefore, the order of 5 in $(\bbZ_{11}^*,\times)$ is $5$.
  }
  %%%%%%%%%%%%%%%%%%
    \vspace{0.09\textheight}
    \item (10 points) Let $x$ be an element in $(\bbZ_p^*,\times)$ such that $x^n=1 \mod{p}$ for some positive integer $n$ and let $h$ be the order of $x$ in $(\bbZ_p^*,\times)$. Prove that $h$ divides $n$.\newline
  %%%%% ANSWER %%%%%
  {\bfseries
      \newline
      First off, we know that $x^n = 1\mod p$ and that $x^h = 1\mod p$. To simplify, we will call $1\mod p = e$, where $e$ is the identity for this group. Therefore, $x^n = e$ and $x^h = e$ \newline
      \newline
      Let us say that $n = c*h + r$, where $c$ is some positive integer coefficient, and $r$ is some remainder. I set it up this way because $h$ is the SMALLEST positive integer such that $x^h = 1\mod p$. Therefore, $n$ will always be greater than or equal to $h$. If we plug in this equation to $x^n = e$ and simplify, we see the following: \newline

      $x^n = e$ \newline
      $x^{ch}*x^r = e$ \newline

      $x^{ch} = e$, so: \newline

      $x^{ch}*x^r = e$ \newline
      $e*x^r = e$ \newline
      $x^r = e$ \newline
      \newline

      We know that $e = 1\mod p$, which will always equal $1$. Therefore, if $x^r = 1$, $r$ has to be $0$. If $r = 0$, that means that $n = c*h$. Clearly, $h$ divides $n$.
  }
  %%%%%%%%%%%%%%%%%%
    \vspace{0.5\textheight}
    \item (5 points) Let $h$ be the order of $x$ in $(\bbZ_p^*,\times)$. Prove that $h$ divides $(p-1)$.\newline
  %%%%% ANSWER %%%%%
  {\bfseries
      \newline
      \newline
      From part C of problem 1, $x^{p-1} = 1\mod p$. Since, in 6b, $x^n = 1\mod p$, we can use the exact same logic as in 6b, but instead drop $p-1$ in the place of $n$, we can clearly see that $h$ divides $(p-1)$.
  }
  %%%%%%%%%%%%%%%%%%
\end{enumerate}
\newpage

%\iffalse
%%%%%%%%%%%%%%%%%%%%%%%%%%%%%%%%%%%%%%%%%%%%%%%%%%%%%%%%%%
%%%%%%%%%%%Problem 7%%%%%%%%%
%%%%%%%%%%%%%%%%%%%%%%%%%%%%%%%%%%%%%%%%%%%%%%%%%%%%%%%%%%
\item {\bfseries Defining Multiplication over $\bbZ_{27}^*$ (25 points).} 
  In the class, we had considered the group $(\bbZ_{26},+)$ to construct a one-time pad for one alphabet messages. 
  A few students were interested in defining a group with 26 elements using a ``multiplication''-like operation. 
  This problem shall assist you to define the $(\bbZ_{27}^*,\times)$ group that has 26 elements.
  
  {\bfseries The first attempt from class.}
  Recall that in the class we had seen that the following is also a group.
    $$ (\bbZ_{27}\setminus\{0,3,6,9,12,15,18,21,24\},\times),$$
  where $\times$ is integer multiplication $\mod 27$. 
  However, the set had only 18 elements. 
  
  In this problem, we shall define $(\bbZ_{27}^*,\times)$ in an alternate manner such that the set has 26 elements. 
  
  {\bfseries A new approach.} 
  Interpret $\bbZ_{27}^*$ as the set of all triplets $(a_0,a_1,a_2)$ such that $a_0,a_1,a_2 \in \bbZ_3$ and at least one of them is non-zero. 
  Intuitively, you can think of the triplets as the ternary representation of the elements in $\bbZ_{27}^*$. 
  We interpret the triplet $(a_0,a_1,a_2)$ as the polynomial $a_0 + a_1X + a_2X^2$. 
  So, every element in $\bbZ_{27}^*$ has an associated non-zero polynomial of degree at most 2, and every non-zero polynomial of degree at most 2 has an element in $\bbZ_{27}^*$ associated with it. 
  
  The multiplication ($\times$ operator) of the element $(a_0,a_1,a_2)$ with the element $(b_0,b_1,b_2)$ is defined as the element corresponding to the polynomial
    $$(a_0 + a_1X + a_2X^2) \times (b_0 + b_1X + b_2X^2) \mod 2 + 2X + X^3$$
    
  The multiplication ($\times$ operator) of the element $(a_0,a_1,a_2)$ with the element $(b_0,b_1,b_2)$ is defined as follows.\newline
  %%%
  \begin{boxedminipage}{\linewidth}
  Input $(a_0,a_1,a_2)$ and $(b_0,b_1,b_2)$.
  \begin{enumerate}
  \item Define $A(X) \defeq a_0 + a_1X + a_2X^2$ and $B(X) \defeq b_0+b_1X+b_2X^2$
  \item Compute $C(X) \defeq A(X)\times B(X)$ (interpret this step as ``multiplication of polynomials with integer coefficients'') 
  \item Compute $R(X) \defeq C(X) \mod 2+2X+X^3$ (interpret this as step as taking a remainder where one treats both polynomials as polynomials with integer coefficients). 
    Let $R(X) = r_0 + r_1X + r_2X^2$
  \item Return $(c_0,c_1,c_2) = (r_0\mod 3, r_1\mod 3, r_2\mod 3)$

  \end{enumerate}
  \end{boxedminipage}
  %%%
  
  
  For example, the multiplication $(0,1,1)\times (1,1,2)$ is computed in the following way.
  \begin{enumerate}
  \item $A(X) = X + X^2$ and $B(X) = 1+X+2X^2$. 
  \item $C(X) = X + 2X^2 + 3X^3 + 2X^4$.
  \item $R(X) = -6-9X-2X^2$.
  \item $(c_0,c_1,c_2) = (0,0,1)$.
  \end{enumerate}
   
  
  According to \underline{this definition} of the $\times$ operator, solve the following problems. 
  \begin{itemize}
  \item (5 points) Evaluate $(1,0,1) \times (1,1,1) $ \newline
  %%%%% ANSWER %%%%%
  {\bfseries
      \newline
      \newline

      \begin{enumerate}
      \item $A(X) = a_0 + a_1X + a_2X^2 = 1 + X^2$.
      \item $B(X) = b_0 + b_1X + b_2X^2 = 1 + X + X^2$
      \item $C(X) = A(X)\times B(X) = (X^2 + 1)(X^2 + X + 1) = 1 + X + 2X^2 + X^3 + X^4$.
      \item $R(X) = C(X) \mod 2+2X+X^3 = (X^4 + X^3 + 2X^2 + X + 1)\mod (2 + 2X + X^3) = -1 -3X$.
      \item $(c_0,c_1,c_2) = (r_0\mod 3, r_1\mod 3, r_2\mod 3) = (2, 0, 0)$.
      \end{enumerate}

      $(1,0,1) \times (1,1,1) = (2,0,0)$
  }
  %%%%%%%%%%%%%%%%%%
    \newpage
  \item (10 points) Note that $e=(1,0,0)$ is a identity element. Find the inverse of $(0,1,1)$.
  \newline
  %%%%% ANSWER %%%%%
  {\bfseries
      \newline
      \newline
      The inverse $i$ of $(0,1,1)$ would mean that $(0,1,1) \times i = (1,0,0)$. \newline
      If we work backwards using the $\times$ operator, we will be able to find out the inverse. \newline
      \begin{enumerate}
        \item $(c_0,c_1,c_2) = (1,0,0)$, and since $(c_0,c_1,c_2)$ is a modulo of $R(X)$, we can set $R(X) = (1,0,0)$, so $R(X) = 1$.
        \item $R(X) = C(X)\mod 2+2X+X^3$, and $C(X) = A(X) \times B(X)$. We know that $A(X) = X+X^2$, so we have to make $(B(X) \times (X+X^2)) \pmod {2+2X+X^3}$ be a polynomial that reduces into $(1,0,0)$ based on the definition of $\times$ in this group.
        \item After testing some triplets in this group, I found that $(2,1,0) = 2+X$ works. By the definition of $\times$ in this group, $(0,1,1) \times (2,1,0) \pmod {2+2X+X^3} = -2+3X^2$. This corresponds to the polynomial $(-2\mod 3,0,3\mod 3)$, which simplifies to $(1,0,0)$.\newline
      \end{enumerate}
      Therefore, the inverse of $(0,1,1) = (2,1,0)$.
  }
  %%%%%%%%%%%%%%%%%%
    \newpage
  
  \item (10 points) Assume that $(\bbZ_{27}^*,\times)$ is a group. 
    Find the order of the element $(1,1,0)$. 
  
    (Recall that, in a group $(G,\circ)$, the order of an element $x\in G$ is the smallest positive integer $h$ such that $\overbrace{x\circ x \circ \dotsi \circ x}^{h\text{-times}} = e$)
    \newline
  %%%%% ANSWER %%%%%
  {\bfseries
      \newline
      \newline
      The element $(1,1,0)$ corresponds to the polynomial $1+X$. We need to find the smallest positive integer $h$ such that $(1+X)^h \mod 2+2X+X^3=e$. \newline

      The result of the equation $(1+X)^{13} \mod 2+2X+X^3 = -3767-7065X-2826X^2$. Turning it into the form $(r_0\mod 3, r_1\mod 3, r_2\mod 3)$, we get the result $(-2826\mod 3, -7065\mod 3, -3767\mod 3) = (1,0,0)$, which means that the order $h$ of the element $(1,1,0)$ is $13$.
  }
  %%%%%%%%%%%%%%%%%%
    \newpage
  \end{itemize}

\end{enumerate}
\newpage


%%%%%%%%%%%%%%%%%%%%%%%%%%%%%%%%%%%%%%%%%%%%%%%%%%
%%%%%%%%%%%% PLEASE LIST COLLABORATORS BELOW  %%%%%
%%%%%%%%%%%%%%%%%%%%%%%%%%%%%%%%%%%%%%%%%%%%%%%%%%%
{\bfseries Collaborators: Vidur Gupta, Giovanni Ordonez} \newline 
% ENTER THEIR NAMES HERE  

\end{document}
