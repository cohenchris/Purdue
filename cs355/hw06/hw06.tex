\documentclass[11pt]{article}
%% Packages
\usepackage[T1]{fontenc}
\usepackage{amssymb}
\usepackage{amsmath}
\usepackage{amsthm}
\usepackage{amsfonts}
\usepackage{mathtools}
\usepackage{xspace}
\usepackage{bm}
\usepackage{nicefrac}
\usepackage{commath}
\usepackage{bbm}
\usepackage{boxedminipage}
\usepackage{xparse}
\usepackage{xcolor}
\usepackage{float}
\usepackage{multirow}
\usepackage{graphicx}
\usepackage{caption}
\usepackage{subcaption}
\usepackage{ifthen}
\usepackage{algpseudocode}
\usepackage{proba} 
\usepackage{colortbl} 
\usepackage{soul} 
%\usepackage{upgreek}
%\usepackage{paralist}
%\usepackage{enumitem}
%\usepackage{times}

\usepackage{tikz}
\usetikzlibrary{positioning}
\usetikzlibrary{fit}
\usetikzlibrary{calc}
\usetikzlibrary{backgrounds}
\usetikzlibrary{shapes}
\usetikzlibrary{patterns}
\usetikzlibrary{matrix}

\usepackage{geometry}
%\usepackage[pdftex,bookmarks=true,pdfstartview=FitH,colorlinks,linkcolor=blue,filecolor=blue,citecolor=blue,urlcolor=blue]{hyperref}
%    \urlstyle{sf}

%\usepackage{enumitem}


%% Document Design Choices
%\setlength{\parindent}{0pt}
%\setlength{\parskip}{10pt}
%\renewcommand{\labelitemi}{\ensuremath{\circ}}
%\geometry{left=1in, right=1in, top=1in, bottom=1in}


%% Marking
\newcommand\TODO[1]{{\color{red}{#1}}\xspace}
\newcommand\change[1]{{\color{red}\underline{#1}}\xspace}
\newcommand{\commentdg}[1]{{\color{blue}{#1}}\xspace}
\newcommand\commentop[1]{{\color{red}{#1}}\xspace}


%% Theorems and Environments
\newtheorem{conjecture}{Conjecture}
%\newtheorem{theorem}{Theorem}
%\newtheorem{importedtheorem}{Imported Theorem}
%\newtheorem{informaltheorem}{Informal Theorem}
%\newtheorem{definition}{Definition}
%\newtheorem{lemma}{Lemma}
\newtheorem{claim}{Claim}
%\newtheorem{corollary}[theorem]{Corollary}
%\newtheorem{observation}{Observation}
%\newtheorem{property}{Property}


\newenvironment{boxedalgo}
  {\begin{center}\begin{boxedminipage}{1\linewidth}}
  {\end{boxedminipage}\end{center}}

\newenvironment{inlinealgo}[1]
  {\noindent\hrulefill\\\noindent{\bf {#1}}}
  {\hrulefill}


%% References
\newcommand{\namedref}[2]{\hyperref[#2]{#1~\ref*{#2}}\xspace}
\newcommand{\lemmaref}[1]{\namedref{Lemma}{lem:#1}}
\newcommand{\theoremref}[1]{\namedref{Theorem}{thm:#1}}
\newcommand{\inftheoremref}[1]{\namedref{Informal Theorem}{infthm:#1}}
\newcommand{\claimref}[1]{\namedref{Claim}{clm:#1}}
\newcommand{\corollaryref}[1]{\namedref{Corollary}{corol:#1}}
\newcommand{\figureref}[1]{\namedref{Figure}{fig:#1}}
\newcommand{\equationref}[1]{\namedref{Equation}{eq:#1}}
\newcommand{\defref}[1]{\namedref{Definition}{def:#1}}
\newcommand{\observationref}[1]{\namedref{Observation}{obs:#1}}
\newcommand{\procedureref}[1]{\namedref{Procedure}{proc:#1}}
\newcommand{\importedtheoremref}[1]{\namedref{Imported Theorem}{impthm:#1}}
\newcommand{\informaltheoremref}[1]{\namedref{Informal Theorem}{infthm:#1}}
\newcommand{\sectionref}[1]{\namedref{Section}{sec:#1}}
\newcommand{\appendixref}[1]{\namedref{Appendix}{app:#1}}
\newcommand{\propertyref}[1]{\namedref{Property}{prop:#1}}


%% English
\newcommand{\ie}{\text{i.e.}\xspace}
%\newcommand{\st}{\text{s.t.}\xspace}
\newcommand{\etal}{\text{et al.}\xspace}
\newcommand{\naive}{\text{na\"ive}\xspace}
\newcommand{\wrt}{\text{w.r.t.}\xspace}
\newcommand{\whp}{\text{w.h.p.}\xspace}
\newcommand{\resp}{\text{resp.}\xspace}
\newcommand{\eg}{\text{e.g.}\xspace}
\newcommand{\cf}{\text{cf.}\xspace}
\newcommand{\visavis}{\text{vis-\`a-vis}\xspace}

%% Authors
\newcommand{\damgard}{Damg{\aa}rd\xspace}


%% Crypto
\def\alice{{{Alice}}\xspace}
\def\bob{{{Bob}}\xspace}
\def\eve{{{Eve}}\xspace}
\newcommand{\poly}{\ensuremath{\mathrm{poly}}\xspace}
\newcommand{\polylog}{\ensuremath{\mathrm{polylog}\;}\xspace}
\newcommand{\negl}{\ensuremath{\mathsf{negl}}\xspace}
\newcommand{\secpar}{\ensuremath{{\lambda}}\xspace}
\newcommand{\zo}{\ensuremath{{\{0,1\}}}\xspace}

  %% Functionalities
  \newcommand{\func}[1]{\ensuremath{\cF_{\mathsf{#1}}}\xspace}
  \newcommand{\fcom}[0]{\func{com}}
  \newcommand{\fot}[0]{\func{ot}}
  
  %% Protocols
  % \newcommand{\prot}[1]{\ensuremath{\pi_{\mathsf{#1}}}\xspace}


%% Graph Theory
\newcommand{\ancest}[0]{\mathsf{Ancestors}}
\newcommand{\sibling}[0]{\mathsf{Siblings}}
\newcommand{\parent}[0]{\mathsf{parent}}
\newcommand{\leaves}[0]{\mathsf{leaves}}


%% Math Symbols
\renewcommand{\o}[0]{\ensuremath{\circ}\xspace}
\newcommand{\dist}[1]{\ensuremath{\left\langle{#1}\right\rangle}\xspace}
%\newcommand{\prob}[2]{\ensuremath{\dist{{#1}_1, \dotsc, {#1}_{#2}}}\xspace}
\newcommand{\ip}[2]{\ensuremath{\left\langle{#1},{#2}\right\rangle}\xspace}
\renewcommand{\vec}[1]{\ensuremath{\mathbf{#1}}\xspace}
\newcommand{\concat}[0]{\ensuremath{\circ}\xspace}
\newcommand{\nin}[0]{\ensuremath{\not\in}\xspace}
\newcommand{\xor}[0]{\ensuremath{\oplus}\xspace}
\newcommand{\rv}[1]{\ensuremath{\mathbf{#1}}\xspace}
\newcommand{\p}[1]{\ensuremath{^{{\left(#1\right)}}}\xspace}
\newcommand{\argmax}[0]{\ensuremath{\mathop{\mathrm{argmax}}~}\xspace}
\newcommand{\argmin}[0]{\ensuremath{\mathop{\mathrm{argmin}}~}\xspace}
\newcommand{\me}{\mathrm{e}}
\NewDocumentCommand\mathstack{>{\SplitList{;}}m}
  {\ensuremath{
    \begin{smallmatrix}
      \ProcessList{#1}{ \insertone }    
    \end{smallmatrix}
  }}
\newcommand{\insertone}[1]{\ensuremath{#1}\\}
\newcommand{\tuple}[1]{\ensuremath{{\left\langle{#1}\right\rangle}}\xspace}

%\newcommand{\argmax}{\operatornamewithlimits{argmax}}
%\DeclareMathOperator*{\argmax}{arg\,max}

  
  %% General
  \newcommand{\ceil}[1]{\ensuremath{\left\lceil{#1}\right\rceil}\xspace}
  \newcommand{\floor}[1]{\ensuremath{\left\lfloor{#1}\right\rfloor}\xspace}
  %\newcommand{\abs}[1]{\ensuremath{\left\vert{#1}\right\vert}\xspace}
  \newcommand{\lone}[1]{\ensuremath{\left\vert{#1}\right\vert}\xspace}
  \newcommand{\spnorm}[1]{\ensuremath{\left\Vert{#1}\right\Vert}\xspace}

  %% Renamed Symbols
  \newcommand{\eps}[0]{\varepsilon}
  \let\epsilon\eps
  %\let\phi\varphi
  
  %% Cal Alphabets
  \newcommand{\cA}{\ensuremath{{\mathcal A}}\xspace}
  \newcommand{\cB}{\ensuremath{{\mathcal B}}\xspace}
  \newcommand{\cC}{\ensuremath{{\mathcal C}}\xspace}
  \newcommand{\cD}{\ensuremath{{\mathcal D}}\xspace}
  \newcommand{\cE}{\ensuremath{{\mathcal E}}\xspace}
  \newcommand{\cF}{\ensuremath{{\mathcal F}}\xspace}
  \newcommand{\cG}{\ensuremath{{\mathcal G}}\xspace}
  \newcommand{\cH}{\ensuremath{{\mathcal H}}\xspace}
  \newcommand{\cI}{\ensuremath{{\mathcal I}}\xspace}
  \newcommand{\cK}{\ensuremath{{\mathcal K}}\xspace}
  \newcommand{\cL}{\ensuremath{{\mathcal L}}\xspace}
  \newcommand{\cM}{\ensuremath{{\mathcal M}}\xspace}
  \newcommand{\cN}{\ensuremath{{\mathcal N}}\xspace}
  \newcommand{\cO}{\ensuremath{{\mathcal O}}\xspace}
  \newcommand{\cP}{\ensuremath{{\mathcal P}}\xspace}
  \newcommand{\cQ}{\ensuremath{{\mathcal Q}}\xspace}
  \newcommand{\cR}{\ensuremath{{\mathcal R}}\xspace}
  \newcommand{\cS}{\ensuremath{{\mathcal S}}\xspace}
  \newcommand{\cT}{\ensuremath{{\mathcal T}}\xspace}
  \newcommand{\cU}{\ensuremath{{\mathcal U}}\xspace}
  \newcommand{\cV}{\ensuremath{{\mathcal V}}\xspace}
  \newcommand{\cW}{\ensuremath{{\mathcal W}}\xspace}
  \newcommand{\cX}{\ensuremath{{\mathcal X}}\xspace}
  \newcommand{\cY}{\ensuremath{{\mathcal Y}}\xspace}
  \newcommand{\cZ}{\ensuremath{{\mathcal Z}}\xspace}
  
  %% Bold Alphabets
  \newcommand{\bA}{\ensuremath{{\mathbf A}}\xspace}
  \newcommand{\bB}{\ensuremath{{\mathbf B}}\xspace}
  \newcommand{\bC}{\ensuremath{{\mathbf C}}\xspace}
  \newcommand{\bD}{\ensuremath{{\mathbf D}}\xspace}
  \newcommand{\bE}{\ensuremath{{\mathbf E}}\xspace}
  \newcommand{\bF}{\ensuremath{{\mathbf F}}\xspace}
  \newcommand{\bG}{\ensuremath{{\mathbf G}}\xspace}
  \newcommand{\bQ}{\ensuremath{{\mathbf Q}}\xspace}
  \newcommand{\bU}{\ensuremath{{\mathbf U}}\xspace}
  \newcommand{\bV}{\ensuremath{{\mathbf V}}\xspace}
  \newcommand{\bX}{\ensuremath{{\mathbf X}}\xspace}
  \newcommand{\bY}{\ensuremath{{\mathbf Y}}\xspace}
  \newcommand{\br}{\ensuremath{{\mathbf r}}\xspace}
  
  %% Black-board Bold Alphabets
  \newcommand{\bbA}{\ensuremath{{\mathbb A}}\xspace}
  \newcommand{\bbB}{\ensuremath{{\mathbb B}}\xspace}
  \newcommand{\bbC}{\ensuremath{{\mathbb C}}\xspace}
  \newcommand{\bbD}{\ensuremath{{\mathbb D}}\xspace}
  \newcommand{\bbE}{\ensuremath{{\mathbb E}}\xspace}
  \newcommand{\bbF}{\ensuremath{{\mathbb F}}\xspace}
  \newcommand{\bbG}{\ensuremath{{\mathbb G}}\xspace}
  \newcommand{\bbH}{\ensuremath{{\mathbb H}}\xspace}
  \newcommand{\bbI}{\ensuremath{{\mathbb I}}\xspace}
  \newcommand{\bbJ}{\ensuremath{{\mathbb J}}\xspace}
  \newcommand{\bbK}{\ensuremath{{\mathbb K}}\xspace}
  \newcommand{\bbN}{\ensuremath{{\mathbb N}}\xspace}
  \newcommand{\bbO}{\ensuremath{{\mathbb O}}\xspace}
  \newcommand{\bbP}{\ensuremath{{\mathbb P}}\xspace}
  \newcommand{\bbQ}{\ensuremath{{\mathbb Q}}\xspace}
  \newcommand{\bbR}{\ensuremath{{\mathbb R}}\xspace}
  \newcommand{\bbS}{\ensuremath{{\mathbb S}}\xspace}
  \newcommand{\bbT}{\ensuremath{{\mathbb T}}\xspace}
  \newcommand{\bbU}{\ensuremath{{\mathbb U}}\xspace}
  \newcommand{\bbV}{\ensuremath{{\mathbb V}}\xspace}
  \newcommand{\bbZ}{\ensuremath{{\mathbb Z}}\xspace}
  
  %% Fraktur Alphabets
  \newcommand{\fP}{\ensuremath{{\mathfrak P}}\xspace}
  \newcommand{\fR}{\ensuremath{{\mathfrak R}}\xspace}
  \newcommand{\fX}{\ensuremath{{\mathfrak X}}\xspace}
  
  %% Overline Aphabets
  \newcommand{\op}{\ensuremath{{\overline p}}\xspace}
  
  %% Hat Aphabets
  \newcommand{\hf}{\ensuremath{{\widehat f}}\xspace}
  \newcommand{\hg}{\ensuremath{{\widehat g}}\xspace}
  
  %% Fractions
  \newcommand{\half}{\ensuremath{\frac12}\xspace}
  
  %% Set
  \newcommand{\comp}[1]{\ensuremath{\overline{{#1}}}\xspace}
  
  %% Models
  \newcommand{\defeq}[0]{\ensuremath{{\;\vcentcolon=\;}}\xspace}
  \newcommand{\eqdef}[0]{\ensuremath{{\;=\vcentcolon\;}}\xspace}
  \newcommand{\entails}[0]{\ensuremath{{\;\models\;}}\xspace}
  
  %% Matrix
  \newcommand{\tran}[0]{\ensuremath{^{\mathsf{T}}}\xspace}
  
  %% Probability and Distributions
  \newcommand{\event}[1]{\ensuremath{\mathsf{#1}}\xspace}
  \newcommand{\supp}[0]{\ensuremath{\mathsf{Supp}}\xspace}
  \let\pr\prob
  %\newcommand{\pr}[0]{\mathop{\mathrm{Pr}}\xspace}
  \newcommand{\var}[0]{\mathop{\mathrm{Var}}\xspace}
  \newcommand{\cov}[0]{\mathop{\mathrm{Cov}}\xspace}
  %\renewcommand{\Pr}[0]{\mathrm{\generateerror}\xspace}
  %\newcommand{\E}[0]{\mathop{\bbE}\xspace}
  \newcommand{\getsr}[0]{\mathbin{\stackrel{\mbox{\,\tiny \$}}{\gets}}}
  \newcommand{\drawn}{\ensuremath{\sim}\xspace}
  %\newcommand{\sd}[2]{\ensuremath{\mathbf{\Delta}\left({#1},{#2}\right)}\xspace}
  \newcommand{\sd}[2]{\ensuremath{\mathrm{SD}\left({#1},{#2}\right)}\xspace}
  \newcommand{\hd}{\ensuremath{\mathrm{HD}}\xspace}
  \newcommand{\kl}[2]{\ensuremath{\mathrm{D_{KL}}\left({#1},{#2}\right)}\xspace}
  \newcommand{\iid}[0]{\text{i.i.d.}\xspace}
  \newcommand{\ent}[0]{\ensuremath{\mathrm{H}}\xspace}
  
    %% Random Variables
    \newcommand{\rvA}{\rv{A}}
    \newcommand{\rvB}{\rv{B}}
    \newcommand{\rvC}{\rv{C}}
    \newcommand{\rvD}{\rv{D}}
    \newcommand{\rvG}{\rv{G}}
    \newcommand{\rvP}{\rv{P}}
    \newcommand{\rvQ}{\rv{Q}}
    \newcommand{\rvR}{\rv{R}}
    \newcommand{\rvS}{\rv{S}}
    \newcommand{\rvT}{\rv{T}}
    \newcommand{\rvU}{\rv{U}}
    \newcommand{\rvV}{\rv{V}}
    \newcommand{\rvW}{\rv{W}}
    \newcommand{\rvX}{\rv{X}}
    \newcommand{\rvY}{\rv{Y}}
    \newcommand{\rvZ}{\rv{Z}}
    
    \newcommand{\rvm}{\rv{m}}
    \newcommand{\rvr}{\rv{r}}
    \newcommand{\rvx}{\rv{x}}
    \newcommand{\rvy}{\rv{y}}
    \newcommand{\rvc}{\rv{c}}

  %% Binary Operators
  \newcommand{\band}[0]{\ensuremath{~\wedge~}\xspace}
  \newcommand{\bor}[0]{\ensuremath{~\vee~}\xspace}
  \let\leq\leqslant
  \let\le\leqslant
  \let\geq\geqslant
  \let\ge\geqslant
  
  %% Combinatorial
  \renewcommand\choose[2]{\ensuremath{{
    \left(\begin{matrix}
      {#1}\\
      {#2}
    \end{matrix}\right)}}\xspace}
  \newcommand\smallchoose[2]{\ensuremath{{
    \left(\begin{smallmatrix}
      {#1}\\
      {#2}
    \end{smallmatrix}\right)}}\xspace}
  \newcommand{\fallfact}[2]{\ensuremath{{#1}_{\left({#2}\right)}}\xspace}
  %\newcommand{\fact}[1]{\ensuremath{{#1}!}\xspace}
  
  %% Set Operations
  \newcommand{\union}[0]{\ensuremath{\cup}\xspace}
  \newcommand{\intersect}[0]{\ensuremath{\cap}\xspace}
  \newcommand{\setdiff}[0]{\ensuremath{\Delta}\xspace}
  

%% Algorithms, Predicates
\newcommand{\pred}[1]{\ensuremath{\mathsf{#1}}\xspace}


%% Local Terms
\newcommand{\mycite}[1]{{\color{brown}{{#1}}}\xspace}
\newcommand{\wt}[0]{\ensuremath{\mathsf{wt}}\xspace}
\newcommand{\Inf}[0]{\ensuremath{\pred{Inf}}\xspace}
%\newcommand{\Var}[0]{\ensuremath{\pred{Var}}\xspace}
\newcommand{\mac}[0]{\pred{Mac}} 
\newcommand{\sk}[0]{\pred{sk}} 
\newcommand{\pk}[0]{\pred{pk}} 
\newcommand{\gen}[0]{\pred{Gen}} 
\newcommand{\enc}[0]{\pred{Enc}} 
\newcommand{\dec}[0]{\pred{Dec}} 
\newcommand{\SK}[0]{\pred{sk}} 
\newcommand{\capprox}[0]{\approx\p c}
\renewcommand{\tag}[0]{\pred{Tag}} 
\newcommand{\sign}[0]{\pred{Sign}} 
\newcommand{\ver}[0]{\pred{Ver}} 
\newcommand{\simu}[0]{\pred{Sim}} 
\newcommand{\ball}[0]{\pred{Ball}\xspace}
\newcommand{\vol}[0]{\pred{Vol}\xspace}


\newcommand{\pout}[0]{\ensuremath{\p{\text{out}}}\xspace}
\newcommand{\pin}[0]{\ensuremath{\p{\text{in}}}\xspace}

\newcommand{\Ext}[0]{\ensuremath{\pred{Ext}}\xspace}
\newcommand{\inv}[0]{\pred{inv}}
\providecommand{\id}[0]{\pred{id}}


\usepackage{fancyhdr}   
\pagestyle{fancy}      
\lhead{CS 355, FALL 2020}               
\rhead{Name: Hemanta K. Maji} %%% <-- REPLACE Hemanta K. Maji WITH YOUR NAME HERE

\usepackage[strict]{changepage}  
\newcommand{\nextoddpage}{\checkoddpage\ifoddpage{\ \newpage\ \newpage}\else{\ \newpage}\fi}  


\begin{document}

\title{Homework 6}

\date{}

\maketitle 

\thispagestyle{fancy}  
\pagestyle{fancy}      


\begin{enumerate}

\item {\bfseries RSA Assumption (5+12+5).} Consider RSA encryption scheme with parameters $N=35=5\times 7$.
\begin{enumerate}
    \item Find $\varphi(N)$ and $\mathbb{Z}^{*}_N$. 
    \newline

    $\mathbb{Z}^{*}_N$ = \{ 1, 2, 3, 4, 6, 7, 9, 11, 12, 13, 16, 17, 18, 19, 22, 23, 24, 26, 27, 29, 31, 32, 33, 34\} \newline

    And since $\varphi(N) = \abs{\mathbb{Z}^{*}_N}$, $\varphi(N) = 24$. \newline

    %\newpage
    \vspace{0.3 \textheight}
    \item Use repeated squaring and complete the rows $X,X^2, X^4$ for all $X\in \mathbb{Z}^{*}_N$ as you have seen in the class (slides), that is, fill in the following table by adding as many columns as needed.

  {\bfseries Solution.}  \newline

   \begin{center}
       \begin{tabular}{|c|c|c|c|c|c|c|c|c|c|c|c|c|}
       \hline
        $X$ & 1 & 2  & 3  & 4  & 6 & 8  & 9  & 11 & 12 & 13 & 16 & 17 \\
       \hline
        $X^2$ & 1 & 4  & 9  & 16 & 1 & 29 & 11 & 16 & 4  & 29 & 11 & 9 \\
        \hline
        $X^4$ & 1 & 16 & 11 & 11 & 1 & 1  & 16 & 11 & 16 & 1  & 16 & 11 \\
       \hline
       \end{tabular}
    \end{center}

      \begin{center}
        \begin{tabular}{|c|c|c|c|c|c|c|c|c|c|c|c|c|}
        \hline
          $X$ & 18 & 19 & 22 & 23 & 24 & 26 & 27 & 29 & 31 & 32 & 33 & 34 \\
        \hline
          $X^2$ & 9  & 11 & 29 & 4  & 16 & 11 & 29 & 1  & 16 & 9  & 4  & 1 \\
        \hline
          $X^4$ & 11 & 16 & 1  & 16 & 11 & 16 & 1  & 1  & 11 & 11 & 16 & 1 \\
        \hline
      \end{tabular}
      \end{center}


   \vspace{0.2 \textheight}
    \item Find the row $X^5$ and show that $X^5$ is a bijection from $\mathbb{Z}^{*}_N$ to $\mathbb{Z}^{*}_N$.
     \newline
  {\bfseries Solution.}  \newline

   \begin{center}
       \begin{tabular}{|c|c|c|c|c|c|c|c|c|c|c|c|c|}
       \hline
        $X$ & 1 & 2  & 3  & 4  & 6 & 8  & 9  & 11 & 12 & 13 & 16 & 17 \\
       \hline
        $X^4$ & 1 & 16 & 11 & 11 & 1 & 1  & 16 & 11 & 16 & 1  & 16 & 11 \\
        \hline
          $X^5$ & 1 & 32 & 33 & 9  & 6 & 8  & 4  & 16 & 17 & 13 & 11 & 12 \\
       \hline
       \end{tabular}
    \end{center}

      \begin{center}
        \begin{tabular}{|c|c|c|c|c|c|c|c|c|c|c|c|c|}
        \hline
          $X$ & 18 & 19 & 22 & 23 & 24 & 26 & 27 & 29 & 31 & 32 & 33 & 34 \\
        \hline
          $X^4$ & 11 & 16 & 1  & 16 & 11 & 16 & 1  & 1  & 11 & 11 & 16 & 1 \\
        \hline
          $X^5$ & 23 & 24 & 22 & 18 & 19 & 31 & 27 & 29 & 26 & 2  & 3  & 34 \\
        \hline
      \end{tabular}
      \end{center}

      $X^5$ is clearly a bijection since every element of X shows up exactly once.
      
    
     \newpage
\end{enumerate}
   

\item {\bfseries Answer to the following questions (7+7+7+7):}
\begin{enumerate}
    \item Compute the three least significant (decimal) digits of $6251007^{1960404}$ by hand.\newline 
  
      \bfseries{
        The three least significant decimal digits of $6251007^{1960404}$ is equivalent to $6251007^{1960404} \pmod{1000}$. We can further reduce the complexity of this problem by observing that $6251007^{1960404} \pmod{1000} = 007^{1960404} \pmod{1000} = 7^{1960404} \pmod{1000}$. \newline

        We learned about Euler's totient in class, and one theorem that will help us in this problem. \newline
        For x and N that are relatively prime, $x^{\varphi(N)} = 1\mod N$ \newline

        $\varphi(1000) = \varphi(2^3 * 5^3) = (2^{3-1}*5^{3-1})(2-1)(5-1) = (2^2 * 5^2)(1)(4) = (100)(1)(400) = 400 \newline
        \varphi(1000) = 400 \newline$

        Therefore, \newline
        $x^{400} = 1\mod 1000$ \newline

        And since \newline
        $1960404 \mod 400 = 4$ \newline

        We can conclude that $7^{1960404} \mod 1000 = 7^4 \mod 1000$.


%        Following the same logic, we can also reduce the exponent. $1950404 \pmod{1000} = 404$, so $7^{1960404} \pmod{1000} = 7^{404} \pmod{1000}$. Now, we can use repeated squares to solve this problem. $\mod 1000$ is abstracted out for simplicity. \newline

        Using repeated squares, we can solve the problem ($\mod 1000$ is abstracted out for simplicity): \newline

        $7^1 = \alpha_0 = 7^{2^0} = 7 \newline
        7^2 = \alpha_1 = 7^{2^1} = \alpha_0 * \alpha_0 = 49 \newline
        7^4 = \alpha_2 = \alpha_1 * \alpha_1 = 49 * 49 = 2401 = 401 \mod 1000 \newline$

        Therefore, the three least significant decimal digits of $6251007^{1960404} = 401$.
      }





    \vspace {0.4\textheight}
    \item Is the following RSA signature scheme valid?(Justify your answer)\newline
    $(r||m)=24, \sigma=196, N=1165, e=43$\newline
    Here, $m$ denotes the message, and $r$ denotes the randomness used to sign $m$ and $\sigma$ denotes the signature. Moreover, $(r||m)$ denotes the concatenation of $r$ and $m$. The signature algorithm $Sign(m)$ returns $(r||m)^d \mod{N}$ where $d$ is the inverse of $e$ modulo $\varphi(N)$. The verification algorithm $Ver(m,\sigma)$ returns ($(r||m)==\sigma^e \mod{N}$). 
    \newline
    {\bfseries

      When we run the $Ver(m, \sigma)$ function, we get the following: \newline
      $Ver(m, \sigma) = ( (r||m) == \sigma^e \mod N) \newline
      = (24 == 196^{43} \mod 1165) \newline
      = (24 == 676) \newline
      = FALSE$ \newline

      Since the verification function fails, this RSA signature scheme is INVALID. \newline
    }
 
    \newpage
    \item 
    Remember that in RSA encryption and signature schemes, $N=p\times q$ where $p$ and $q$ are two large primes. Show that in a RSA scheme (with public parameters $N$ and $e$), if you know $N$ and $\varphi(N)$, then you can find the factorization of $N$ i.e. you can find $p$ and $q$.
    \newline
   {\bfseries
    From class, we know that: \newline
    $N = pq \newline
    \varphi(N) = (p-1)(q-1) \newline
    $

    Simplifying, $\varphi(N) = pq-p-q+1$. Notice that we can substitute $N=pq$ into that equation. Simplifying further, \newline

    $\varphi(N) = N-p-q+1 \newline
    p+q = N - \varphi(N) + 1 \newline$

    And since the entire right side of that final equation is known, $p$ and $q$ become much easier to find. \newline
   }

    \vspace {0.1\textheight}
    \item Consider an encryption scheme where 
    $Enc(m):=m^e \mod{N}$ where $e$ is a positive integer relatively prime to $\varphi(N)$ and $Dec(c):=c^d \mod{N}$ where $d$ is the inverse of $e$ modulo $\varphi(N)$.
    Show that in this encryption scheme, if you know the encryption of $m_1$ and the encryption of $m_2$, then you can find 
    the encryption of $(m_1\times m_2)^5$.
    \newline 
   {\bfseries

    First, we know that: \newline
    $(x\mod N * y\mod N)\mod N = xy\mod N$ \newline

    Therefore, if we set the encryption of $m_1 = x$ and the encryption of $m_2 = y$, we can get the encryption of $(m1 \times m2)$ by calculating $(x*y)\mod N$. To clearly illustrate: \newline

    $((m_1^e\mod N)*(m_2^e\mod N))\mod N = (m_1 * m_2)^e \mod N \newline$ \newline

    Therefore, we can find the encryption of $(m_1 \times m_2)^5$ by doing the following: \newline

    $((m_1^e\mod N)^5*(m_2^e\mod N)^5)\mod N \newline$

    $= ((m_1^{5e}\mod N)*(m_2^{5e}\mod N))\mod N \newline$

    $= (m_1 * m_2)^{5e} \mod N \newline$ \newline
   }
\end{enumerate}
    \newpage
%%%%%%%%%%%%%%%%%%%%%%%%%%%%%%%%%%%%%%%%%%%%%%%%%%
%%%%%%%%%%%% PROGRAMMING ASSIGNMENT  %%%%%%%%%%%%%
%%%%%%%%%%%%%%%%%%%%%%%%%%%%%%%%%%%%%%%%%%%%%%%%%%%


\item {\bfseries Programming Assignment: Compute the Cube Root of a Large Integer.} (50 points)\newline
This problem requires you to find the cube-root of very large perfect cube integers (each number is roughly 30K bits in binary representation). 
The inputs shall be given using a text file \textsf{inputs.txt} containing five (roughly) 30K-bit numbers in binary representation.
These numbers are separated by a new line character. 
Your program must output the cube roots of the five numbers, represented as binary and separated by a new line character, to a text file named as \textsf{outputs.txt}.
Make sure that you follow the conventions, otherwise you will get zero credit. 
You can use Java, Python, C, or SageMath. 
Turn in your code and the output file via Gradescope.
%In this programming assignment, you are asking to write a program that computes the cube root of a very large (roughly 30K bits in binary representation) perfect cube number using binary search algorithm. Since the given number is a perfect cube, it is guaranteed that its cube root is an integer. You will be given a text file named \textit{inputs.txt} containing five (roughly) 30K-bit numbers in binary representation. These numbers are separated by new line characters.  Your program must output the cube roots of the five numbers, represented as binary and separated by new line characters, to a text file named as \textit{yourLastName-yourFirstName.txt}. Make sure that you follow the naming convention. You can use Java, Python, C, or SageMath. Turn in your code and the output file. 
\end{enumerate}
%%%%%%%%%%%%%%%%%%%%%%%%%%%%%%%%%%%%%%%%%%%%%%%%%%
%%%%%%%%%%%% PLEASE LIST COLLABORATORS BELOW  %%%%%
%%%%%%%%%%%%%%%%%%%%%%%%%%%%%%%%%%%%%%%%%%%%%%%%%%%
\newpage
{\bfseries Collaborators:} \newline 
% ENTER THEIR NAMES HERE  

\end{document}

